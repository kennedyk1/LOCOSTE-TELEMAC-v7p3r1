%-------------------------------------------------------------------------------
\chapter[Text and equations example]{Text and equations example and extra text
to make this a very long chapter title that we would not want to appear in
whole in the header}
%-------------------------------------------------------------------------------

%-------------------------------------------------------------------------------
\section{First standard section}
%-------------------------------------------------------------------------------

%...............................................................................
\subsection{Title of your first subsection}
%...............................................................................

Lorem ipsum dolor sit amet, consectetur adipisicing elit, sed do eiusmod tempor
incididunt ut labore et dolore magna aliqua. Ut enim ad minim veniam, quis
nostrud exercitation ullamco laboris nisi ut aliquip ex ea commodo consequat.
Duis aute irure dolor in reprehenderit in voluptate velit esse cillum dolore eu
fugiat nulla pariatur. Excepteur sint occaecat cupidatat non proident, sunt in
culpa qui officia deserunt mollit anim id est laborum.

Lorem ipsum dolor sit amet, consectetur adipisicing elit, sed do eiusmod tempor
incididunt ut labore et dolore magna aliqua. Ut enim ad minim veniam, quis
nostrud exercitation ullamco laboris nisi ut aliquip ex ea commodo consequat.
Duis aute irure dolor in reprehenderit in voluptate velit esse cillum dolore eu
fugiat nulla pariatur. Excepteur sint occaecat cupidatat non proident, sunt in
culpa qui officia deserunt mollit anim id est laborum.

%...............................................................................
\subsection{Title of your second subsection}
%...............................................................................

Lorem ipsum dolor sit amet, consectetur adipisicing elit, sed do eiusmod tempor
incididunt ut labore et dolore magna aliqua. Ut enim ad minim veniam, quis
nostrud exercitation ullamco laboris nisi ut aliquip ex ea commodo consequat.
Duis aute irure dolor in reprehenderit in voluptate velit esse cillum dolore eu
fugiat nulla pariatur. Excepteur sint occaecat cupidatat non proident, sunt in
culpa qui officia deserunt mollit anim id est laborum.

%...............................................................................
\subsubsection{Title of your first subsubsection}
%...............................................................................

Lorem ipsum dolor sit amet, consectetur adipisicing elit, sed do eiusmod tempor
incididunt ut labore et dolore magna aliqua. Ut enim ad minim veniam, quis
nostrud exercitation ullamco laboris nisi ut aliquip ex ea commodo consequat.
Duis aute irure dolor in reprehenderit in voluptate velit esse cillum dolore eu
fugiat nulla pariatur. Excepteur sint occaecat cupidatat non proident, sunt in
culpa qui officia deserunt mollit anim id est laborum.

%...............................................................................
\subsubsection{Title of your second subsubsection}
%...............................................................................

Lorem ipsum dolor sit amet, consectetur adipisicing elit, sed do eiusmod tempor
incididunt ut labore et dolore magna aliqua. Ut enim ad minim veniam, quis
nostrud exercitation ullamco laboris nisi ut aliquip ex ea commodo consequat.
Duis aute irure dolor in reprehenderit in voluptate velit esse cillum dolore eu
fugiat nulla pariatur. Excepteur sint occaecat cupidatat non proident, sunt in
culpa qui officia deserunt mollit anim id est laborum.

%-------------------------------------------------------------------------------
\section[Short section title]{This is a very long section title that we would
not want to appear in whole in the header}
%-------------------------------------------------------------------------------

Lorem ipsum dolor sit amet, consectetur adipisicing elit, sed do eiusmod tempor
incididunt ut labore et dolore magna aliqua. Ut enim ad minim veniam, quis
nostrud exercitation ullamco laboris nisi ut aliquip ex ea commodo consequat.
Duis aute irure dolor in reprehenderit in voluptate velit esse cillum dolore eu
fugiat nulla pariatur. Excepteur sint occaecat cupidatat non proident, sunt in
culpa qui officia deserunt mollit anim id est laborum.

Lorem ipsum dolor sit amet, consectetur adipisicing elit, sed do eiusmod tempor
incididunt ut labore et dolore magna aliqua. Ut enim ad minim veniam, quis
nostrud exercitation ullamco laboris nisi ut aliquip ex ea commodo consequat.
Duis aute irure dolor in reprehenderit in voluptate velit esse cillum dolore eu
fugiat nulla pariatur. Excepteur sint occaecat cupidatat non proident, sunt in
culpa qui officia deserunt mollit anim id est laborum.

%-------------------------------------------------------------------------------
\section{Example of equations}
%-------------------------------------------------------------------------------

%...............................................................................
\subsection{One line equation}
%...............................................................................

Here is how you should write your equations:

\begin{align}
A=B+C
\label{eq:FirstEquation}
\end{align}

Here is how you should refer to the equation; i.e. see
equation~\ref{eq:FirstEquation}. However if you want you can also refer to them
as eqn.~\eqref{eq:FirstEquation}, just be consistent.

Always compile twice to have the links appear in the PDF.

%...............................................................................
\subsection{Multiple line equation}
%...............................................................................

Here is how you should write very long equations:

\begin{align}
First Long Argument &=	 One long variable
\notag \\
					&\quad + Another long variable
% or
%	 				&\phantom{{}={}} + Another long variable
\label{eqn:LongEquation}
\end{align}

Here is how you should write connected equations:

\begin{subequations}
\label{eq:MultipleEquations}
\begin{align}
a &= b+c
\\
d &= e+f+g
\end{align}
\end{subequations}

Or if you want to give them labels:

\begin{subequations}
\label{eq:LabeledMultipleEquations}
\begin{align}
\text{first equation:}&&
a &= b+c
&&
\\
\text{second equation:}&&
d &= e+f+g
&&
\label{eq:LabeledMultipleEquations2}
\end{align}
\end{subequations}

To reference them it is easier to refer to them as: see
equations~\ref{eq:MultipleEquations} or see
eqn.~\eqref{eq:LabeledMultipleEquations2}.

%...............................................................................
\subsection{Equations using the \texorpdfstring{\telemacsystem{}}{Telemac}
format}
%...............................................................................

The vectors need to be defined using \verb+\vec{}+, which gives:

\begin{align}
\vec{A} &= \vec{B}
\end{align}

For the operators, the appropriate commands need to be used as well (eg:
\verb+\Grad+, \verb+\Div+ or \verb+\Lap+):

\begin{align}
\Grad(A_b) &= \text{Gradient}
\\
\Div(A_b) &= \text{Divergence}
\\
\Lap(A_b) &= \text{Laplacian}
\end{align}


%-------------------------------------------------------------------------------
\section{Formatting the \texorpdfstring{\texttt{*.tex}}{tex} file}
%-------------------------------------------------------------------------------

It is recommended to have a \texttt{*.tex} file per chapter, and to keep them
in a seperate folder. In the \texttt{*.tex} file, it is recommended to put
commented lines to above and below section and subscetion names, i.e.:

\begin{scriptsize}
\begin{verbatim}
%-------------------------------------------------------------------------------
\chapter{Text and equations example}
%-------------------------------------------------------------------------------

%-------------------------------------------------------------------------------
\section{First standard section}
%-------------------------------------------------------------------------------

%...............................................................................
\subsection{Title of your first subsection}
%...............................................................................
\end{verbatim}
\end{scriptsize}

Also the text in the file should be justified to 80 characters.

%...............................................................................
\subsection{Naming Convention}
%...............................................................................

Names of files, or variables should be written in CamelCase; i.e. without
spaces, underscore or dash, and a capital at the start of each word.

%...............................................................................
\subsection{Referenced pointers}
%...............................................................................

When referencing pointers the label should have the type of object referenced
in the name, and the following naming convention is suggested:

\begin{itemize}
\item Chapters: \verb+\label{ch:ChapterLabel}+
\item Sections or subsections: \verb+\label{se:SectionLabel}+
\item Equations: \verb+\label{eq:EquationLabel}+
\item Figures: \verb+\label{fig:EquationLabel}+
\item Tables: \verb+\label{tab:EquationLabel}+
\end{itemize}


%-------------------------------------------------------------------------------
\section{\texorpdfstring{\telemacsystem{}}{Telemac} specific commands}
%-------------------------------------------------------------------------------

Shortcuts are given for all the modules, i.e. \telemacsystem{}, \artemis{},
\bief{}, \sisyphe{}, \telemac{2D}, \telemac{3D} and \tomawac{}. As a reminder,
when using the commands, brackets should be used afterwards to ensure that the
spacing is correrctly defined; i.e. \verb+\telemacsystem{}+.

