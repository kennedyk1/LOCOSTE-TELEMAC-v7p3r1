%----------------------------------------------------------------------------------------------------
\chapter{Documentation}
%----------------------------------------------------------------------------------------------------

This chapter will describe how the documentation is handled in \telemacsystem.
All the documentation files are written in \LaTeX and are using the same Style.

%----------------------------------------------------------------------------------------------------
\section{The structure}
%----------------------------------------------------------------------------------------------------
All the documentation files can be found in the "documentation" folder at the
root of the \telemacsystem sources. The first level as a folder for each module
plus a Misc Folder which contains all the documentation that are either generic
to the while system or not link to a module.

Then for each module we have the following folders:
\begin{itemize}
\item reference
\item release\_note
\item user
\item validation
\end{itemize}

Each documentation folder contains those files/folder:
\begin{verbatim}
main.tex
graphics
-- image.png
latex
-- file.tex
\end{verbatim}

Where :
\begin{itemize}
\item main.tex The main \LaTeX file.
\item graphics Folder contains all the figures associated with the
  documentation.
\item latex Contains all the files that create the documentation.
\end{itemize}

In \telfile{Misc/TelemacTemplateDoc} you can find a template that describes how
to write a \telemacsystem documentation.

\begin{WarningBlock}{Warning:}
The reference documentation is automatically generated from the dictionary of
the module so all the modification must be made in the dictionary file.
\end{WarningBlock}

\begin{WarningBlock}{Warning:}
The validation manual merge all the latex contains in the \telfile{doc} folder
of the test cases.
\end{WarningBlock}

%----------------------------------------------------------------------------------------------------
\section{How to compile}
%----------------------------------------------------------------------------------------------------

Everything is done using the python script docTELEMAC.py. By default it will
compile all the documentation. By adding "-m module" it will compile only the
documentation for that specific module, by adding "-M miscdoc" it will only
compile the documentation "miscdoc" from the \telfile{Misc} folder. And the
options "--reference,--user..." compile only the reference,user...
documentation.

%----------------------------------------------------------------------------------------------------
\section{How to convert a Word documentation into LaTeX using GrindEQ}
%----------------------------------------------------------------------------------------------------

The main conversion word is made using the software GrindEQ.

Before running the conversion it is strongly suggested to remove the index.

After the conversion is ran all that is left is some search and replace to
comply with the template.

Here are a couple of things you will encounter:
\begin{itemize}
\item All lot of figure will be created as for a weird reason some letter are
represented with pictures. To have the \LaTeX file compile I suggest using a
dummy picture and copy it to the according name (it should be
image1,image2\ldots).
\item  Remove all the \verb!\!
\item All the equations should compile directly but you should reorganise them
in the \LaTeX so they are more readable.
\end{itemize}
