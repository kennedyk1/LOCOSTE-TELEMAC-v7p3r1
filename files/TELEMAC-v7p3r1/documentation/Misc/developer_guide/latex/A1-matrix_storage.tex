\chapter{Matrix storage conventions}

The off-diagonal terms of a matrix A are placed in a two-dimensional array, the
first dimension being NELMAX, the maximum number of elements (the second
dimension depends on the type of matrix and storage). The following conventions
are written in the form of rows, such that:

\begin{lstlisting}[language=TelFortran]
XA(IELEM,01) = XA12
\end{lstlisting}

This row means the following:

The contribution of element IELEM to the coefficient of point 2 in the equation
for point 1 is placed in XA(IELEM,01). 1 and 2 refer to the local numbering of
element IELEM, and so the general number of point 1 is IKLE(IELEM,1).

The location is also given in the form of a matrix whose elements give the
second dimension of XA in which the corresponding term is placed. These
matrices are indeed used in \bief, in which they are given in the form of DATA
tables. In view of FORTRAN placing notation for two-dimensional arrays, the
matrices appear in transposed form. For this reason, the transposed form of the
matrices is also given here.

\section{Triangle P1-P1 EBE}

$\left(\begin{array}{ccc}
    {-} & {1} & {2} \\
    {4} & {-} & {3} \\
    {5} & {6} & {-}
\end{array}\right)$
transposed form:
$\left(\begin{array}{ccc}
    {-} & {4} & {5} \\
    {1} & {-} & {6} \\
    {2} & {3} & {-}
\end{array}\right)$


\begin{lstlisting}[language=TelFortran]
XA(IELEM,1) = XA12
XA(IELEM,2) = XA13
XA(IELEM,3) = XA23
\end{lstlisting}

If A is asymmetrical:

\begin{lstlisting}[language=TelFortran]
XA(IELEM,04) = XA21
XA(IELEM,05) = XA31
XA(IELEM,06) = XA32
\end{lstlisting}

\section{Triangle P1-Quasi-bubble EBE}

$\left(\begin{array}{cccc}
    {-} & {1} & {2} & {3} \\
    {4} & {-} & {5} & {6} \\
    {7} & {8} & {-} & {9}
\end{array}\right)$
transposed form:
$\left(\begin{array}{ccc}
    {-} & {4} & {7} \\
    {1} & {-} & {8} \\
    {2} & {5} & {-} \\
    {3} & {6} & {9} \\
\end{array}\right)$

\begin{lstlisting}[language=TelFortran]
XA(IELEM,01) = XA12
XA(IELEM,02) = XA13
XA(IELEM,03) = XA14
XA(IELEM,04) = XA21
XA(IELEM,05) = XA23
XA(IELEM,06) = XA24
XA(IELEM,07) = XA31
XA(IELEM,08) = XA32
XA(IELEM,09) = XA34
\end{lstlisting}

\section{Triangle Quasi-bubble-P1 EBE}

$\left(\begin{array}{ccc}
    {-} & {1} & {2} \\
    {3} & {-} & {4} \\
    {5} & {6} & {-} \\
    {7} & {8} & {9} \\
\end{array}\right)$
transposed form:
$\left(\begin{array}{cccc}
    {-} & {3} & {5} & {7} \\
    {1} & {-} & {6} & {8} \\
    {2} & {4} & {-} & {9}
\end{array}\right)$

\begin{lstlisting}[language=TelFortran]
XA(IELEM,01) = XA12
XA(IELEM,02) = XA13
XA(IELEM,03) = XA21
XA(IELEM,04) = XA23
XA(IELEM,05) = XA31
XA(IELEM,06) = XA32
XA(IELEM,07) = XA41
XA(IELEM,08) = XA42
XA(IELEM,09) = XA43
\end{lstlisting}

\section{Triangle Quasi-bubble Quasi-bubble or Quadrilateral Q1-Q1 EBE}

$\left(\begin{array}{cccc}
    {-} & {1} & {2} & {3} \\
    {7} & {-} & {4} & {5} \\
    {8} & {10} & {-} & {6} \\
    {9} & {11} & {12} & {-}
\end{array}\right)$
transposed form~:
$\left(\begin{array}{cccc}
    {-} & {7} & {8} & {9} \\
    {1} & {-} & {10} & {11} \\
    {2} & {4} & {-} & {12} \\
    {3} & {5} & {6} & {-}
\end{array}\right)$

\begin{lstlisting}[language=TelFortran]
XA(IELEM,01) = XA12
XA(IELEM,02) = XA13
XA(IELEM,03) = XA14
XA(IELEM,04) = XA23
XA(IELEM,05) = XA24
XA(IELEM,06) = XA34
\end{lstlisting}

If A is asymmetrical:

\begin{lstlisting}[language=TelFortran]
XA(IELEM,07) = XA21
XA(IELEM,08) = XA31
XA(IELEM,09) = XA41
XA(IELEM,10) = XA32
XA(IELEM,11) = XA42
XA(IELEM,12) = XA43
\end{lstlisting}


\section{Triangle P1 EBE - Quadratic triangle EBE}

\[\left(\begin{array}{cccccc}
      {-} & {1} & {2} & {3} & {4} & {5} \\
      {6} & {-} & {7} & {8} & {9} & {10} \\
      {11} & {12} & {-} & {13} & {14} & {15}
\end{array}\right) \]

\begin{lstlisting}[language=TelFortran]
XA(IELEM,01) = XA12
XA(IELEM,02) = XA13
XA(IELEM,03) = XA14
XA(IELEM,04) = XA15
XA(IELEM,05) = XA16
XA(IELEM,06) = XA21
XA(IELEM,07) = XA23
XA(IELEM,08) = XA24
XA(IELEM,09) = XA25
XA(IELEM,10) = XA26
XA(IELEM,11) = XA31
XA(IELEM,12) = XA32
XA(IELEM,13) = XA34
XA(IELEM,14) = XA35
XA(IELEM,15) = XA36
\end{lstlisting}

\section{Quadratic triangle EBE - Triangle P1 EBE}

\[\left(\begin{array}{ccc}
      {-} & {1} & {2} \\
      {3} & {-} & {4} \\
      {5} & {6} & {-} \\
      {7} & {8} & {9} \\
      {10} & {11} & {12} \\
      {13} & {14} & {15}
\end{array}\right) \]

\begin{lstlisting}[language=TelFortran]
XA(IELEM,01) = XA12
XA(IELEM,02) = XA13
XA(IELEM,03) = XA21
XA(IELEM,04) = XA23
XA(IELEM,05) = XA31
XA(IELEM,06) = XA32
XA(IELEM,07) = XA41
XA(IELEM,08) = XA42
XA(IELEM,09) = XA43
XA(IELEM,10) = XA51
XA(IELEM,11) = XA52
XA(IELEM,12) = XA53
XA(IELEM,13) = XA61
XA(IELEM,14) = XA62
XA(IELEM,15) = XA63
\end{lstlisting}

\section{Prism P1-P1 EBE or Quadratic triangle EBE}

\[\left(\begin{array}{cccccc}
      {-}  & {1} & {2} & {3} & {4} & {5} \\
      {16} & {-} & {6} & {7} & {8} & {9} \\
      {17} & {21} & {-} & {10} & {11} & {12} \\
      {18} & {22} & {25} & {-} & {13} & {14} \\
      {19} & {23} & {26} & {28} & {-} & {15} \\
      {20} & {24} & {27} & {29} & {30} & {-} \\
\end{array}\right) \]

transposed form:

\[\left(\begin{array}{cccccc}
      {-}  & {16} & {17} & {18} & {19} & {20} \\
      {1} & {-} & {21} & {22} & {23} & {24} \\
      {2} & {6} & {-} & {25} & {26} & {27} \\
      {3} & {7} & {10} & {-} & {28} & {29} \\
      {4} & {8} & {11} & {13} & {-} & {30} \\
      {5} & {9} & {12} & {14} & {15} & {-} \\
\end{array}\right) \]

\begin{lstlisting}[language=TelFortran]
XA(IELEM,01) = XA12
XA(IELEM,02) = XA13
XA(IELEM,03) = XA14
XA(IELEM,04) = XA15
XA(IELEM,05) = XA16
XA(IELEM,06) = XA23
XA(IELEM,07) = XA24
XA(IELEM,08) = XA25
XA(IELEM,09) = XA26
XA(IELEM,10) = XA34
XA(IELEM,11) = XA35
XA(IELEM,12) = XA36
XA(IELEM,13) = XA45
XA(IELEM,14) = XA46
XA(IELEM,15) = XA56
\end{lstlisting}

If A is asymmetrical:
\begin{lstlisting}[language=TelFortran]
XA(IELEM,16) = XA21
XA(IELEM,17) = XA31
XA(IELEM,18) = XA41
XA(IELEM,19) = XA51
XA(IELEM,20) = XA61
XA(IELEM,21) = XA32
XA(IELEM,22) = XA42
XA(IELEM,23) = XA52
XA(IELEM,24) = XA62
XA(IELEM,25) = XA43
XA(IELEM,26) = XA53
XA(IELEM,27) = XA63
XA(IELEM,28) = XA54
XA(IELEM,29) = XA64
XA(IELEM,30) = XA65
\end{lstlisting}
