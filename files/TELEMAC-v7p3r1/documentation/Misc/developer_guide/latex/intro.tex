%
%
\chapter{Introduction}
%
%
%
\section{Why a finite element library?}
%
A great many finite element software programs have been developed over the last
few years at the Laboratoire National d'Hydraulique et Environnement. These
have been based on a single data structure, initiated by the development of
\telemac{2D}. Algorithms used by one program, for example to process a diffusion
operator, can also be used by another. It was therefore felt to be quite
natural to group together all the numerical developments of the various codes
in a single library, distinguishing them from the physical aspects.
\\
This finite element library (called \bief in the rest of this document. \bief
stands for the French expression \textbf{BI}blioth\`{e}que d'\textbf{E}lements
\textbf{F}inis, meaning Finite Element Library, but ``bief'' in French is a
river reach in English) is designed so that it can be used as a toolbox in the
simplest possible way. It is possible, for example, to solve a classic fluid
mechanics equation by calling the \bief \textit{ad hoc} modules, without having
to worry about the details of the solution. This simplifies and considerably
speeds up the calculation code development phase. In addition, \bief continually
includes new developments, thereby making them available to users immediately.
\\
The development of \bief is closely linked to that of the TELEMAC system codes,
most of which are the subject of a Quality Control procedure. In the case of
the software programs of the EDF's Research and Development, this procedure
involves designing and then checking the quality of the product throughout the
different phases in its life. In particular, a software program subjected to
Quality Control is accompanied by a validation document which describes a
series of test cases. This document can be used to evaluate the qualities and
limitations of the product and identify its field of application. These test
cases are also used for developing the software and are checked each time a
modification is made. Consequently, the \bief algorithms also benefit from this
strict quality control procedure.
%
\section{Brief description of \bief}

The data structure and programming of \bief is described in detail in Part~B of
this document. One of the important features of this structure is that matrices
are stored either in elementary form or in an edge-by-edge storage. Compared
with compact storage, this type of storage saves on memory space for numerous
types of elements and also enables resolution algorithms to be obtained quickly
and efficiently. In fact, one of the essential features of \bief is to offer
methods with very low computing costs.
\\
\bief offers a whole range of subroutines. They include methods for solving
advection equations, diffusion equations, linear system inversion methods with
different types of preconditioning. The user is also provided with subroutines
for calculating matrices: mass matrices, diffusion matrices, boundary matrices,
etc. \bief can be used to carry out all the conventional operations on vectors
(norms, dot products, etc.), on the products of one matrix by a vector or of
two matrices. By simply calling a subroutine, the user can calculate the
divergence of a vector, the gradient of a function, and so on. It should be
remembered that this description is not exhaustive and that the content of \bief
will change depending on the requirements of its users.
\\
The language used is FORTRAN 90 (see explanations below), and, to facilitate
the diffusion of the TELEMAC system, portability is checked on a wide range of
hardware, including both super-computers and workstations, Linux and Windows
machines.

\section{Fortran 90: reasons for the change}

TELEMAC was written in FORTRAN 77 up to version 3.2. There are a number of
reasons for the choice of FORTRAN 90 for \bief:
\begin{itemize}
\item Structured programming. Structures were prepared in version 3.2 of \bief
  in FORTRAN 77, at the price of non-standard features, for example the use of
  negative indices in arrays. The structures are now normal structures in
  FORTRAN 90, and they are much easier to use.
\item Dynamic allocation of memory.
\item The increasing number of modules based on \bief meant that it was taking
  longer and longer to complete each update. One of the aim of structured
  programming is to simplify updating.
\item The increasing number of arguments in the subroutines, and changes of
  arguments in the user subroutines. This will is suppressed as far as possible
  by the use of FORTRAN 90 modules.
\end{itemize}
The principal objectives of structured programming are therefore:
\begin{itemize}
\item To enable dynamic allocation of memory.
\item To facilitate update and development of the system elements.
\item To facilitate the future development and maintenance of \bief.
\item To get a safer implementation, with many error checking done by the
  compiler itself.
\item To enable a better compatibility between subsequent versions of \bief.
\end{itemize}
%
%



