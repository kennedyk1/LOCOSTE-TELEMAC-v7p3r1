\chapter{ Getting Started with TelApy  module}
\label{ch:inp:outp}

\section{TelApy module installation}

In order to be able to use TelApy module, the \telemacsystem and all its
external libraries must be compiled in dynamic form. The explanation of dynamic
compilation is available on the \telemacsystem website in the wiki category
"installation notes 2"
(\url{http://wiki.opentelemac.org/doku.php?id=installation_notes_2_beta}).

Then after compiling the module, the use of TelApy is presented and explained
in some notebooks documentation. In fact, the TelApy module is provided with
some tutorial intended for people who want to run \telemac{2D} in an
interactive mode with the help of the \python programming language.

\section{How to run notebook documentation}

In order to read notebooks, the user needs to install a notebook viewer such as
jupyter notebook. Notebook documents (or "notebooks", all lower case) are
documents which contain both computer code (e.g. \python) and rich text elements
(paragraph, equations, figures, links, etc...). Notebook documents are both
human-readable documents containing the analysis description and the results
(figures, tables, etc..) as well as executable documents which can be run to
perform data analysis.

First and foremost, the Jupyter Notebook is an interactive environment for
writing and running code. The notebook is capable of running code in a wide
range of languages. However, each notebook is associated with a single kernel.
This notebook is associated with the I\python kernel, therefore runs \python
code. More details on the installation and use can be found in the Jupyter
website \url{http://jupyter.org}.

\section{Notebook examples in TelApy}

As already mentioned, the TelApy module is provided with some tutorial intended
for people who want to run \telemac{2D} in an interactive mode with the help of
the \python programming language. Currently, these tutorials based on notebook
examples are located in the folder
$\$HOMETEL/scripts/python27/TelApy/notebooks$. In this directory, two notebooks
categories are available:

\begin{itemize}
\item the folder $api$ contains tutorials to run \telemac{2D} in an interactive
  mode with the help of the \python programming language. The interactive mode
  means that communication with \telemac{2D} becomes possible throughout the
  simulation without having to stop it. One can easily set or get the value of
  any variables at each time step with the help of special communication
  functions: the Application Programming Interfaces (API).
  \begin{itemize}
  \item "telemac2d.ipynb" is a tutorial dedicated to the API description and
    the presentation of \telemac{2D} \python functionalities.
   \item "telemac2d\_example.ipynb" is a tutorial showing a computation run of
     \telemac{2D} in an interactive mode with the help of the \python programming
     language.
  \end{itemize}
\item  The folder $optim$ is dedicated, as indicated by his name, to
  optimization problem.
  \begin{itemize}
  \item "genop.ipynb" is a tutorial intended for people who want to use the
    Genop optimizer. Genop (Genetic optimizer) is a \python package implementing
    the Genetic Algorithm (GA) for a mono-objective minimization. GA is a
    derivative-free optimizer. This metaheuristic mimics the natural evolution
    with the repeated application of operators (selection, mutation, crossover,
    etc.) in order to evolve a set of solutions towards the optimality. People
    interested in this class of algorithms may refer to Genetic Algorithms or
    Genetic Programming for more information.
  \item "telemac2d\_optim\_genop.ipynb" is a tutorial showing how to optimize a
    \telemac{2D} case with a genetic algorithm based on Genop \python module.
    \item "telemac2d\_optim\_newop.ipynb" is a tutorial showing how to optimize
      a \telemac{2D} case with the deterministic algorithm based on the SciPy
      package.
  \end{itemize}
\end{itemize}

In order to see and run the notebook examples, the user need to launch the
command "jupyter notebook example\_name.ipynb". This will launch a Jupyter
server and a page should appear in your default internet browser.
