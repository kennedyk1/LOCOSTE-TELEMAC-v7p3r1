%--------------------------------------------------------------------------------
\chapter{Bathymetry treatment}
%--------------------------------------------------------------------------------
The mesh file on \telemacsystem standard can hold a bahymetric information (or
bottom topography) on each point of the mesh. This information can be generated
by \stbtel thanks to two sources:
\begin{itemize}
\item Bottom topography files
\item The mesh generator file itself
\end{itemize}
Furthermore, when the assessment of the memory space necessary for making the
computation is done, \stbtel considers that the number of bathymetric points
does not go beyond 20 000. In the opposite, the user must give the right value
with the keyword \telkey{MAXIMUM NUMBER OF BATHYMETRIC POINTS}.
%--------------------------------------------------------------------------------
\section{Use of the bottom topography files}
%--------------------------------------------------------------------------------
Using any kind of mesh generator  (also while a Selafin file is read), \stbtel
can interpolate on the treated mesh a bathymetry, product of one or more data
files. Their names must be given by the user with the keyword \telkey{BOTTOM
TOPOGRAPHY FILES}. \stbtel can manage five bottom topography files. When using
several files, the user  must  check  the absence of common areas between the
datas or their coherence.\\
Algorithm used by \stbtel to interpolate the bottom topography is the
following: for each mesh point, the space is divided into four quadrants
(depends on the horizontal and vertical). For each quadrant, the software
identifies the nearest bathymetry point, then a balance is done using the found
points. Near the boundary, or, once again, on the island case, it could be
important to ignore the points too close from the boundary. Indeed, the
information found on these points could not be considered. The user is helped
with the keyword \telkey{MINIMUM DISTANCE AT BOUNDARY}. When the mesh point are
interpolated, this keyword dimensions the minimal distance below which a
bathymetry points should be ignored.\\
If the \stbtel informations are too short to interpolate the bathymetry on a
point, the value given to the point is automatically 10-6 m and an error
message is printed on the listing printout.

%--------------------------------------------------------------------------------
\section{Use of the universal file}
%--------------------------------------------------------------------------------
When using a TRIGRID or FASTTABS file, the user can decide to recover in the
mesh file the topography informations . He uses the logical keyword
\telkey{BATHYMETRY ON THE UNIVERSAL FILE} (default value NO).\\
A special information is used with TRIGRID. Indeed, the software works on an
information like "level of water", but not on an information like “bottom
level”. When treating the mesh by \stbtel, it is necessary to restore the
correct bottom elevation value. This computation is done with the help of the
formula: Zf = -Ht + CORR where Zf is the bottom elevation value written by
\stbtel in the geometry file, Ht  the value of the level of water read on the
TRIGRID file, and where CORR  is the user specified value with the keyword
\telkey{BOTTOM CORRECTION OF TRIGRID}. The value of this keyword depends on the
convention chosen by the user when he entered the bathymetric informations in
TRIGRID. Usually, it is the same value used with the utility program \verb!sin2tri!
for the translation of a SINUSX file SINUSX to a TRIGRID file.
