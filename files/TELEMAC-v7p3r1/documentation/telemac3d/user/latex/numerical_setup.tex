\chapter{Numerical setup of the computation}

The numerical setup is comparatively common to a hydrodynamic computation alone
or with a tracer. Thus, in the following sections of this chapter, the
numerical parameters as applied to the solution of a tracer equation are
integrated into the hydrodynamic parameters.

\section{General setup}

\telemac{3D} solves the Navier-Stokes equations in several stages, possibly
through the three stages of the fractional step method (see the theoretical
note). The first stage consists in finding out the advected velocity components
by only solving the convection terms of the momentum equations. The second
stage computes, from the advected velocities, the new velocity components by
taking into account both diffusion and source terms of the momentum equations.
These two solutions enable to get an intermediate velocity field. The third
stage computes the water depth from the vertical integration of the continuity
equation and momentum equations only including the pressure-continuity terms.
This step is called the propagation step.

The user can activate or deactivate, either globally or individually, some of
these stages.


\subsection{Advection step}
\label{sec:advstep}
Whether the convection terms will be considered or not will be determined by
means of the logical keyword \telkey{ADVECTION STEP} (default
value YES). However, even though that keyword is set to YES, then some
advection terms can be deactivated by means of the following complete keywords
(value 0 = "NO ADVECTION "):

\begin{itemize}
\item \telkey{SCHEME FOR ADVECTION OF VELOCITIES}: for the advection of
velocities,

\item \telkey{SCHEME FOR ADVECTION OF DEPTH}: for taking the advection of
depth into account,

\item \telkey{SCHEME FOR ADVECTION OF K-EPSILON}: for the advection of power
and turbulent dissipation,

\item \telkey{SCHEME FOR ADVECTION OF TRACERS}: for the advection of tracers
(one value per tracer).
\end{itemize}

See section \ref{sec:advection} for more information on the possible choices.

\subsection{Diffusion step}
\label{sec:difstep}

%Deleted keyword in v7.3
%Whether the diffusion terms are taken into account or not is established by
%means of the logical keyword \telkey{DIFFUSION STEP} (default value YES).
%However, even though that keyword is set to YES, then
Some diffusion terms can be deactivated by means of the following complete
keywords by the following keywords:

\begin{itemize}
\item \telkey{SCHEME FOR DIFFUSION OF VELOCITIES},

\item \telkey{SCHEME FOR DIFFUSION OF TRACERS} (one value common to all tracers),

\item \telkey{SCHEME FOR DIFFUSION OF K-EPSILON}.
\end{itemize}

The 0 value at each keyword cancels the diffusion, whereas the 1 value
(default value) leads to the implicit calculation of diffusion.

%Deleted keyword in v7.3
%For the treatment of the diffusion, two choices are possible for the keyword
%\telkey{OPTION FOR THE DIFFUSION}. The default value 1 means an implicit
%treatment whereas the value 2 means an uncoupled treatment between the
%horizontal diffusion and the vertical diffusion.


\subsection{Propagation step}

%Deleted keyword in v7.3
%The velocity and water depth propagation processes are taken into account by
%the logical keyword \telkey{PROPAGATION STEP} (default value YES).

Since the version 6.0 of TELEMAC, the propagation step (for velocity and water
depth) is necessarily treated with \telemac{2D} "wave equation" option.

The propagation step can be linearized by activating the keyword
\telkey{LINEARIZED PROPAGATION} (default value = NO) particularly when one
conducts a case study for which an analytical solution in the linearized case
is available. It is then necessary to set the water depth around which the
linearization is performed by means of the keyword \telkey{MEAN DEPTH FOR
LINEARIZATION} (default value 0).

With the "wave equation" option, the keyword \telkey{FREE SURFACE COMPATIBILITY
GRADIENT} can be used. A value lower than 1. (which is the default value) makes
it possible to delete the spurious oscillations of the free surface, but
slightly alters the consistency between the water depth and the velocities in
the continuity equation.

When using the non-hydrostatic version, it is possible to enable the dynamic
pressure term in treatment of the wave equation. This option is enabled by the
logical keyword \telkey{DYNAMIC PRESSURE IN WAVE EQUATION} but it leads a
dilemma:

\begin{itemize}
\item If the keyword is set to YES, the dynamic pressure gradient is taken
into account when calculating the evolution of the water depth (advantage), but
the evolution of the water depth is not known when calculating the dynamic
pressure (disadvantage),

\item If the keyword is set to NO (which is the default value), the dynamic
pressure is calculated taking into account the evolution of the water depth
(advantage) but it is not taken into account in the calculation of the water
depth (disadvantage).
\end{itemize}

Therefore, when the effect of the dynamic pressure is important (nonlinear
waves), it is recommended to set this keyword to YES in combination with
setting sub-iterations for nonlinearities (see test case NonLinearWave).


\section{The advection scheme}
\label{sec:advection}
The procedure for taking the advection terms into account is individualized
for each of the variables liable to be processed. It has previously been
explained that the zero option corresponds to a deactivation of the term.


\subsection{Advection of the three-dimensional variables}

The advection schemes of the three-dimensional variables (i.e.
all the variables except the water depth) are (for further details about these
options, the reader shall refer to the Theoretical Note):

\begin{itemize}
\item 0: Deactivation,

\item 1: Method of characteristics. That method involves mutually independent
advection and diffusion steps. The method consists in writing that the value of
the advected variable is equal to the value of the same variable in the
previous instant traced back on the path travelled during the time step,

\item 2: Explicit scheme + SUPG (Streamline Upwind Petrov Galerkin). That
method uses test functions which are deformed in the direction of the current
for the variational method,

\item 3: Explicit Leo Postma scheme,

\item 4: Explicit scheme + MURD (Multidimensional Upwind Residual
Distribution) N scheme,

\item 5: Explicit scheme + MURD PSI scheme,

\item 13: Explicit Leo Postma scheme for tidal flats,

\item 14: Explicit scheme + MURD (Multidimensional Upwind Residual
Distribution) N scheme for tidal flats.
\end{itemize}

The latter five schemes are primarily recommended for the tracers, since they
are advantageous in being conservative and monotonic, i.e. they do not generate
any numerical oscillation. On the other hand, they are more diffusive than
SUPG. In that respect, scheme 5 is an improvement to scheme 4, being less
diffusive perpendicularly to the flow but quite obviously somewhat more
computation time-consuming. Nevertheless, it is still globally less
time-consuming than SUPG.

Distributive schemes (options 3, 4, 5, 13 or 14) are schemes whose stability is
conditioned by a Courant number less than 1. When using one of these schemes,
at each time step, \telemac{3D} performs a test for checking the Courant number
point by point. In case it exceeds the value 1, \telemac{3D} will automatically
execute sub-iterations to satisfy the stability criterion. However, if the
number of sub-iterations exceeds 100, \telemac{3D} considers that the treatment
of the advection term is problematic and the calculation is interrupted,
printing an error message in the list control.

The default value for both velocity and $k$-$\epsilon$ is 1. That value
is advisable, since it is satisfactory in many instances and is by far the
fastest. On the contrary, the default value for the tracers is 5. It is the
most reliable scheme, since the "mass" conservation of the active tracers is a
frequently essential point in \telemac{3D}.

Since version 6.0, the value concerning the water depth advection scheme is
ignored by \telemac{3D}. The optimum advection scheme is automatically selected
by the software (conservative scheme).

According to the schemes used, the mass conservation can be improved by
performing sub-iterations. That consists in a updating, for one given time
step, both advective field and propagative field during several sub-iterations.
Upon the first sub-iteration, the field of velocities is yielded by the results
achieved during the previous time steps. Through that procedure, the
non-linearities can be taken into account in a better way and the mass
conservation can be significantly improved in the cases of schemes 2 and 3. The
number of sub-iterations is set by the keyword \telkey{NUMBER OF SUB ITERATIONS
FOR NON LINEARITIES}, the default value of which is 1 (also refer to the
Theoretical Note).

The SUPG scheme can be configured using specific keywords (see \ref{sec:supg}).

The MURD schemes present several options described in \ref{sec:MURD}.

\subsection{Configuration of the SUPG scheme}
\label{sec:supg}
When using the SUPG method, the user must determine the type of upwinding
desired using the keyword \telkey{SUPG OPTION} which is an array of
4 integers related, to the velocity, the water depth, tracers and
$k$-$\varepsilon $ model respectively.

The possible values are:

\begin{itemize}
\item 0: no upwinding,

\item 1: upwinding with the conventional SUPG method, that is to say, the
upwinding is 1,

\item 2: upwinding with the modified SUPG method, that is to say, the
upwinding is equal to the Courant number.
\end{itemize}

In theory, option 2 is more accurate when the Courant number is less than 1,
but should not be used when the latter is important. Thus, option 2 should be
used in models for which the Courant number remains low. If the Courant number
cannot be estimated, it is strongly recommended to use option 1 (which can be
considered as the most universal).

\subsection{Configuration of the MURD schemes}
\label{sec:MURD}
The N and PSI residual distribution schemes present different options that allow
to increase the accuracy of the numerical scheme or to deal with tidal flats.
The keyword \telkey{SCHEME OPTION FOR ADVECTION OF VELOCITIES} (which can be
found also for $k$-$\epsilon$ and tracers) specifies the desired option and must
be used coupled with the keyword
\telkey{SCHEME FOR ADVECTION OF VELOCITIES}= 4 (or 5).
It can be set to:
\begin{itemize}
 \item 1: explicit scheme (default value),
 \item 2: first order predictor-corrector scheme,
 \item 3: second order predictor-corrector scheme,
 \item 4: locally semi-implicit predictor-corrector scheme
(for tidal flats): LIPS.
\end{itemize}
Like in \telemac{2D}, the predictor-corrector schemes need an additional
parameter which represents the number of iterations for every time step
(or sub-time step) to converge to the solution.
The keyword \telkey{NUMBER OF CORRECTIONS OF DISTRIBUTIVE SCHEMES} plays this
role and it is useful for unsteady cases.
For quasi-steady flows, the number of corrections does not have a large impact
on the solution, so it can be set to 0.
On the other hand, for unsteady flows, it is suggested to set the keyword
\telkey{NUMBER OF CORRECTIONS OF DISTRIBUTIVE SCHEMES}= 2 (at least),
which is a good compromise between accuracy and computational time.
Indeed, by increasing the number of corrections, the scheme is more accurate but
the CPU time rapidly increases.\\
The keyword \telkey{NUMBER OF SUB-STEPS OF DISTRIBUTIVE SCHEMES} can be
activated only for locally semi-implicit predictor-corrector schemes (LIPS).
As the keyword mentions, it allows to subdivide the time step given by the user
in the steering file, into several sub-steps.
Again, it produces an effect on the precision of the scheme and it is convenient
to set this keyword in order to have Courant numbers not too large (around 1).

\subsection{Configuration of the weak characteristics}

When choosing the method of characteristics, two forms can be used with the
keyword \telkey{OPTION FOR CHARACTERISTICS}:

\begin{itemize}
\item 1: the strong form (by default),

\item 2: the weak form.
\end{itemize}

None of them are recommended for the advection of tracers because they are not
mass conservative. The weak form will decrease the diffusion. If the keyword
\telkey{MASS-LUMPING FOR WEAK CHARACTERISTICS} = 1. (default value = 0. i.e. no
mass-lumping), monotonicity of the scheme appears. This weak form should be
more conservative than the strong form. The \telkey{NUMBER OF GAUSS POINTS FOR
WEAK CHARACTERISTICS} defines the number of Gauss points used to compute the
weak characteristics. Possible choices are 1, 3 (default value) and 6. The
bigger the number is, the more conservative the scheme is, but the higher the
computational costs are.


\section{Specific parameters in the non-hydrostatic version}

The application of the software \telkey{NON-HYDROSTATIC VERSION} (default = NO)
requires that complementary keywords be defined.

An equation for vertical velocity is initially solved in a same way as the
$U$ and $V$ components. That equation is only written with the
hydrostatic pressure which, under that hypothesis, is cancelled out with the
gravity term. The convection scheme in that equation is identical to that
chosen for $U$ and $V$ (keyword \telkey{SCHEME FOR ADVECTION OF
VELOCITIES}). The solution of the linear system (for the diffusion step
integrating the advection terms or not) is managed by the various following
keywords:

\begin{itemize}
\item \telkey{SOLVER FOR VERTICAL VELOCITY} refer to subsection \ref{sec:solver} below,

\item \telkey{MAXIMUM NUMBER OF ITERATIONS FOR VERTICAL VELOCITY} refer to
subsection \ref{sec:accuracy} below,

\item \telkey{ACCURACY FOR VERTICAL VELOCITY} refer to subsection \ref{sec:accuracy} below,

\item \telkey{PRECONDITIONING FOR VERTICAL VELOCITY} refer to subsection \ref{sec:precond}
below.
\end{itemize}

Afterwards, \telemac{3D} solves a Poisson equation for the dynamic pressure. The
dynamic pressure gradient plays the part of a correction providing the required
zero divergence on velocity. The solution of the linear system in that equation
is managed by the following keywords:

\begin{itemize}
\item \telkey{SOLVER FOR PPE} refer to subsection \ref{sec:solver} below,

\item \telkey{MAXIMUM NUMBER OF ITERATIONS FOR PPE} refer to subsection \ref{sec:accuracy}
below,

\item \telkey{ACCURACY FOR PPE} refer to subsection \ref{sec:accuracy} below,

\item \telkey{OPTION OF SOLVER FOR PPE} refer to subsection \ref{sec:solver} below,

\item \telkey{PRECONDITIONING FOR PPE} refer to subsection \ref{sec:precond} below.
\end{itemize}

Once that pressure is computed, all the velocity components are updated with
the dynamic pressure gradient which will ensure the zero divergence condition.
Updating that "solenoidal" velocity does not require that a linear system be
solved.

\section{Implicitation}

Apart from the terms of the time derivative, the unknowns $f$ (the velocity and
water depth components) can be considered in both extreme instants $t^{n}$ (the
equation is then referred to as explicit) or $t^{n+1}$ (the equation is then
referred to as implicit). Strictly speaking, and for a 2-order solution in
time, an approach consists in considering the terms in the intermediate
instant$(f^{n} +f^{n+1})/2$. Practically, the latter approach is unstable and
it becomes necessary to define an implicitation coefficient for which the
unknowns are actually discretized in time in the following form:
\[\theta f^{n+1} +(1-\theta )f^{n} .\]
The implicitation coefficients are theoretically always higher than 0.5 (0.55
or 0.6 will generally yield good results).

The user can use the keyword \telkey{IMPLICITATION FOR VELOCITIES} (default
value: 1.) which defines the value of the $\theta_{u}$ coefficient for the
velocity components. The keyword \telkey{IMPLICITATION FOR DEPTH} (default
value: 0.55) is provided to set the value of "propagation height" multiplying
coefficient $\theta_{h}$ Lastly, in order to make the construction of the
various numerical schemes more versatile, a diffusion-specific coefficient
$\theta_{u}^{d}$ is provided (keyword \telkey{IMPLICITATION FOR DIFFUSION}
(default value: 1.) and can be different from $\theta_{u}$).


\section{Solution of the linear systems}

Both discretization and variational formulation of the equations lead to a
linear system which now has to be solved. The direct solution methods are often
not suitable and are overly expensive as soon as there are many unknowns.
Thus, the main ways developed in \telemac{3D} consist in solving the linear
systems by means of iterative solvers. Nevertheless for specific applications,
a direct solver can be used.


\subsection{Solvers}
\label{sec:solver}
According to the relevant numerical parameters, various linear systems are
liable to be solved. The solver used for solving one of these systems can be
selected by the user through the following keywords:

\begin{itemize}
\item \telkey{SOLVER FOR DIFFUSION OF VELOCITIES} (default value: 1),

\item \telkey{SOLVER FOR PROPAGATION} (default value: 1),

\item \telkey{SOLVER FOR PPE} (default value: 1),

\item \telkey{SOLVER FOR VERTICAL VELOCITY} (default value: 1),

\item \telkey{SOLVER FOR DIFFUSION OF TRACERS} (default value: 1, one value for
each tracer),

\item \telkey{SOLVER FOR DIFFUSION OF K-EPSILON }(default value: 1).
\end{itemize}

Each of these keywords can assume a value ranging from 1 to 8, which values
correspond to the following possibilities:

\begin{itemize}
\item 1: conjugate gradient method (when the matrix of the system to solve
is symmetric),

\item 2: conjugate residual method,

\item 3: conjugate gradient method on a normal equation,

\item 4: minimum error method,

\item 5: square conjugate gradient method,

\item 6: CGSTAB (stabilized conjugate gradient) method,

\item 7: GMRES (Generalised Minimum RESidual) method,

\item 8: direct solver.
\end{itemize}

The GMRES method is well suited for improperly conditioned systems. This method
requires that the dimension of the Krylov space be defined. That parameter is
set by means of the following keywords:

\begin{itemize}
\item \telkey{OPTION OF SOLVER FOR DIFFUSION OF VELOCITIES},

\item \telkey{OPTION OF SOLVER FOR PROPAGATION},

\item \telkey{OPTION OF SOLVER FOR PPE},

\item \telkey{OPTION OF SOLVER FOR DIFFUSION OF TRACERS},

\item \telkey{OPTION OF SOLVER FOR DIFFUSION OF K-EPSILON}.
\end{itemize}

The default values are set to 3. The larger that parameter is, the higher the
memory requirements and the number of matrix-vector products per iteration (and
consequently the computational time as well) are, but the better the
convergence is.

\subsection{Accuracies}
\label{sec:accuracy}
The principle of iterative methods consists in getting gradually closer to the
exact solution during the iterations. The systems to be solved imply a relative
accuracy within a range from $10^{-4}$ to $10^{-10}$ with a restricted
number of iterations. Both accuracy and maximum number of iterations should be
set for each system.

Accuracy is specified by the following keywords:

\begin{itemize}
\item \telkey{ACCURACY FOR DIFFUSION OF VELOCITIES} (default value: 1.E-5),

\item \telkey{ACCURACY FOR PROPAGATION} (default value: 1.E-6),

\item \telkey{ACCURACY FOR PPE} (default value: 1.E-4),

\item \telkey{ACCURACY FOR VERTICAL VELOCITY} (default value: 1.E-6),

\item \telkey{ACCURACY FOR DIFFUSION OF TRACERS} (default value: 1.E-6),

\item \telkey{ACCURACY FOR DIFFUSION OF K-EPSILON} (default value: 1.E-6).
\end{itemize}

The maximum number of iterations is specified by the following keywords:

\begin{itemize}
\item \telkey{MAXIMUM NUMBER OF ITERATIONS FOR DIFFUSION OF VELOCITIES
}(default value: 60),

\item \telkey{MAXIMUM NUMBER OF ITERATIONS FOR PROPAGATION} (default value:
200),

\item \telkey{MAXIMUM NUMBER OF ITERATIONS FOR PPE} (default value: 100),

\item \telkey{MAXIMUM NUMBER OF ITERATIONS FOR VERTICAL VELOCITY} (default
value: 100),

\item \telkey{MAXIMUM NUMBER OF ITERATIONS FOR DIFFUSION OF TRACERS }(default
value: 60),

\item \telkey{MAXIMUM NUMBER OF ITERATIONS FOR DIFFUSION OF K-EPSILON }(default
value: 200),

\item \telkey{MAXIMUM NUMBER OF ITERATIONS FOR ADVECTION SCHEMES }(default
value: 10) only for schemes 13 and 14.
\end{itemize}

The user automatically gets information about the solvers upon each listing
printout. The information provided in the listing can be of two types:

\begin{itemize}
\item Either the treatment converged before reaching the maximum allowable
number of iterations, and then \telemac{3D} will provide the number of actually
performed iterations, as well as the achieved accuracy,

\item Or the treatment did not converge early enough. \telemac{3D} will then
provide the message "\telfile{MAXIMUM NUMBER OF ITERATIONS IS REACHED}" and give
the actually achieved accuracy. In some cases, and if the maximum number of
iterations is already set to a high value (for instance over 100), convergence
can then be improved by reducing the time step or, quite often, by improving
the quality of the mesh.
\end{itemize}


\subsection{Preconditionings}
\label{sec:precond}
The iterative methods are sensitive to the "conditioning" of matrices, so that
a complementary preconditioning is necessary in order to reduce the number of
iterations for getting a prescribed accuracy.

\telemac{3D} offers several opportunities for preconditioning. The selection is
made by means of the following keywords:

\begin{itemize}
\item \telkey{PRECONDITIONING FOR DIFFUSION OF VELOCITIES},

\item \telkey{PRECONDITIONING FOR PROPAGATION},

\item \telkey{PRECONDITIONING FOR PPE},

\item \telkey{PRECONDITIONING FOR VERTICAL VELOCITY},

\item \telkey{PRECONDITIONING FOR DIFFUSION OF TRACERS} (one value for each
tracer),

\item \telkey{PRECONDITIONING FOR DIFFUSION OF K-EPSILON}.
\end{itemize}

The available options are:

\begin{itemize}
\item 0:  no preconditioning,

\item 2:  diagonal preconditioning,

\item 3:  diagonal preconditioning with the condensed matrix,

\item 5:  diagonal preconditioning with absolute values,

\item 7:  Crout preconditioning per elementCrout (downgraded in parallel),

\item 11: Gauss-Seidel preconditioning per element (downgraded in parallel),

\item 13: preconditioning matrix is provided by the user,

\item 14: cumulated diagonal preconditioning and Crout preconditioning per
element,

\item 17:  preconditioning through direct solution along each vertical
direction,

\item 21:  cumulated diagonal preconditioning with the condensed matrix and
Crout preconditioning per element,

\item 34:  cumulated diagonal preconditioning with direct solution along each
vertical direction.
\end{itemize}

The default value is 2 for all the preconditionings. Some preconditionings can
be cumulated, namely the diagonal preconditionings with other ones. Since the
basic values are prime numbers, a couple of preconditionings are cumulated by
giving to the keyword the value of the product of the two preconditionings
which one wants to cumulate (e.g. numbers 14 and 21).

\section{Tidal flats}

\telemac{3D} offers several treatment options as regards the tidal areas.

First, if the user has ascertained that his/her model has no tidal area
throughout the simulation, the processing of such areas can be deactivated by
setting the keywords \telkey{TIDAL FLATS} to NO (the default value is YES).
That option makes it possible to save computational time (by deleting the tidal
flat testing).

The tidal flats can be treated in two different ways:

\begin{itemize}
\item In the first case, the equations are treated all over the domain and in
a thorough way. The tidal areas are detected and such terms as the free surface
gradient (in the absence of water, the free surface gradient becomes the bottom
gradient and generates spurious motive terms) are corrected in them,

\item In the second case, the tidal areas are withdrawn from the computation.
The exposed elements are always part of the mesh, but all their contributions
to the computations are cancelled by a so-called "masking" array. Thus, the
data structure and the computations remain formally unchanged, to within the
masking coefficient. That method, however, raises issues as regards the mass
conservation and the exposure and coverage dynamics.
\end{itemize}

The treatment will be selected by means of the keyword \telkey{OPTION FOR THE
TREATMENT OF TIDAL FLATS} which can be set either to 1 or 2, the default value
being 1.

The treatment of negative depths can be specified using the keyword
\telkey{TREATMENT OF NEGATIVE DEPTHS}. A value of 1 (default), consists in a
conservative smoothing of negative depths. The second option is to limit the
flux between the elements to ensure strictly positive water depths. This second
option should be used with advection schemes consistent with tidal flats (+
mass lumping option for height at value 1.). The value 0 means that no special
treatment is performed.

The keyword \telkey{MINIMAL VALUE FOR DEPTH} the default value of which is
-1,000 enables to set the threshold below which the smoothing is. For example,
\telkey{MINIMAL VALUE FOR DEPTH} set to 0.01 means the minimum depth is 1~cm.

The following three keywords are for setting, after coverage, the value of the
variable which has been masked:

\begin{itemize}
\item \telkey{TREATMENT ON TIDAL FLATS FOR VELOCITIES},

\item \telkey{TREATMENT ON TIDAL FLATS FOR TRACERS},

\item \telkey{TREATMENT ON TIDAL FLATS FOR K-EPSILON}.
\end{itemize}

The available options for these keywords are:

\begin{itemize}
\item 0: that option corresponds to a setting to zero of the variable on the
element (default value),

\item 1: that option sets to its prior-to-masking value.
\end{itemize}

The keyword \telkey{THRESHOLD FOR VISCOSITY CORRECTION ON TIDAL FLATS} (default
value is 0.2) allows to specify the minimum water depth from which the
viscosity is gradually reduced (see programming within the \telfile{VISCLIP}
subroutine).

When the three-dimensional mesh has crushed levels (null water depth or fixed
level "hitting" the bottom), it is recommended to activate a specific treatment
that prevents the transfer of very small amounts of water at the calculation
points which have no volume (this situation also tends to degrade the mass
conservation when using distributives PSI and N schemes). This algorithm is
activated with the keyword logical \telkey{BYPASS VOID VOLUMES}. When using PSI
and N schemes compatible with tidal flats, the option is automatically enabled,
even if the keyword is set to NO.

\section{Hydrostatic inconsistencies}

Hydrostatic inconsistencies (linked to the truncature errors in the computation
of the buoyancy terms) are liable to occur on the nearly-zero volume prisms.
The keyword \telkey{HYDROSTATIC INCONSISTENCY FILTER} (default = NO) is
provided for the forces caused by the spurious horizontal pressure gradients
and the various diffusion coefficients on those prisms where at least one of
the lower base nodes has a higher elevation than one of the upper base nodes.

\section{Other parameters}

\subsection{Mass-lumping}

Upon the solution of the linearized system, \telemac{3D} makes it possible to
perform a mass-lumping on the mass matrices. That procedure consists in partly
or wholly returning the mass matrix to its diagonal and enables to
substantially shorten the computational times. The resulting solutions,
however, become smoothed, except for in steady flow conditions in which they
are unchanged. The mass-lumping rate is set by means of the keywords
\telkey{MASS-LUMPING FOR DEPTH}, \telkey{MASS-LUMPING FOR VELOCITIES},
\telkey{MASS-LUMPING FOR DIFFUSION} and \telkey{MASS-LUMPING FOR WEAK
CHARACTERISTICS}. Value 1. means maximum mass-lumping (the mass matrices are
diagonal), value 0. (default value) corresponds to the normal treatment without
any mass-lumping. For further details, the reader shall refer to the \telemac{3D}
Theoretical Note.

\subsection{Convergence aid}

Another way to speed up the system convergence when solving the propagation
step consists in acting upon the initial solution rather than the matrix
proper. To that purpose, the initial value being set for $h$ (actually,
the unknown $h^{n+1}$ is replaced by the incrementation $\delta h=h^{n+1} -h^{n}$)
is modified at the beginning of computation, the user can take action at the
keyword \telkey{INITIAL GUESS FOR DEPTH} which can assume the following values:

\begin{itemize}
\item 0: the initial value of $\delta h=h^{n+1} -h^{n} $ is zero,

\item 1: the initial value of $\delta h$ is equal to the value of $\delta h$
at the previous time step (default value),

\item 2: $\delta h=2h^{n} -\delta h^{n-1} $ where $\delta h^{n} $ is the value
of $\delta h$ at the previous time step, and $\delta h^{n-1} $ is the value of
$\delta h$ two time steps earlier. It is actually an extrapolation.
\end{itemize}


\subsection{Matrix storage}

\telemac{3D} provides a couple of procedures for storing the various matrices it
has to handle, namely the conventional EBE (Element By Element) method and the
segment-wise storage. The latter procedure is faster (by approx 20\%) in most
cases.

The choice among these two types of storage is made by means of the keyword
\telkey{MATRIX STORAGE} which can assume the following values:

\begin{itemize}
\item 1: classical EBE method,

\item 3: edge-based storage (default value).
\end{itemize}

\subsection{Velocities projection}

At the end of the time loop, it is possible to do a treatment which aim is to
cancel the component of the normal velocity at the bottom or the normal
velocity on the solid lateral walls. This check is activated with the logical
keywords \telkey{VELOCITY PROJECTED ON SOLID LATERAL BOUNDARIES} and
\telkey{VELOCITY PROJECTED ON BOTTOM}.

These two options are activated by default.
