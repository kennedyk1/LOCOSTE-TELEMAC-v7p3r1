The original method \cite{thompson} uses the theory of
characteristics, linearised in a direction normal to the boundary,
in the framework of the Saint-Venant equations. Here the depth-averaged
velocity is denoted by $\vec{U}$, with components $U$ and $V$:
\begin{equation}
\left\{\begin{array}{l}
U = \dfrac{1}{h}\displaystyle{\int_b^\eta}u dz \smallskip \\
V = \dfrac{1}{h}\displaystyle{\int_b^\eta}v dz \smallskip
\end{array}\right.
\end{equation}

\textit{A detailed explanation of the original technique:} \smallskip

We explain here in more detail what is said in Reference \cite{hervouet007} page
105 to 108. We neglect diffusion and start from the conservative form of
Saint-Venant equations, put in the following form taken from Reference \cite%
{hervouet007} at page 31, using the fact that the free surface $\eta$ is the
bottom topography plus the depth $h$:

\begin{equation}\label{svt}
\left\{\begin{array}{l}
\dfrac{\partial h}{\partial t}+\Div(h\vec{U})=Sce \medskip \\
\dfrac{\partial (hU)}{\partial t}+\dfrac{\partial }{\partial x}(hUU+g\dfrac{%
h^{2}}{2})+\dfrac{\partial }{\partial y}(hUV)=-gh\dfrac{\partial b}{%
\partial x}+hF_{x} \medskip \\
\dfrac{\partial (hV)}{\partial t}+\dfrac{\partial }{\partial x}(hUV)+\dfrac{%
\partial }{\partial y}(hVV+g\dfrac{h^{2}}{2})=-gh\dfrac{\partial b}{%
\partial y}+hF_{y}  
\end{array}\right.
\end{equation}

Let $F$, $G_{x}$, $G_{y}$ and $S(F)$ be:

\begin{equation*}
F=\left( 
\begin{array}{c}
h \\ 
hU \\ 
hV%
\end{array}
\right)
\end{equation*}

\begin{equation*}
G_{x}=\left( 
\begin{array}{c}
hU \\ 
hU^{2}+g\dfrac{h^{2}}{2} \\ 
hUV%
\end{array}
\right) \text{ and }G_{y}=\left( 
\begin{array}{c}
hV \\ 
hUV \\ 
hV^{2}+g\dfrac{h^{2}}{2}%
\end{array}
\right)
\end{equation*}

\begin{equation*}
S(F)=\left( 
\begin{array}{c}
Sce \smallskip\\ 
-gh\dfrac{\partial b}{\partial x}+hF_{x} \smallskip\\ 
-gh\dfrac{\partial b}{\partial y}+hF_{y}%
\end{array}
\right)
\end{equation*}

The Saint-Venant equations \eqref{svt} can then be written in the following form:

\begin{equation*}
\dfrac{\partial F}{\partial t}+\dfrac{\partial G_{x}}{\partial x}+\dfrac{%
\partial G_{y}}{\partial y}=S(F)
\end{equation*}

The Thompson method as implemented so far in Telemac consists in considering
a local system of coordinates based on a local normal vector $%
\vec{n}$ (normal to the boundary) and a local tangent vector $%
\vec{t}$. If the new system of coordinates is denoted $\xi $ and 
$\zeta $, we have:

\begin{equation*}
\vec{n}=\left( 
\begin{array}{c}
\dfrac{\partial \xi }{\partial x} \smallskip\\ 
\dfrac{\partial \xi }{\partial y}%
\end{array}
\right) ~\text{and~}\vec{t}=\left( 
\begin{array}{c}
\dfrac{\partial \zeta }{\partial x} \smallskip\\ 
\dfrac{\partial \zeta }{\partial y}%
\end{array}
\right) =\left( 
\begin{array}{c}
-\dfrac{\partial \xi }{\partial y} \smallskip\\ 
\dfrac{\partial \xi }{\partial x}%
\end{array}
\right)
\end{equation*}

We keep these notations here, but the directions $\vec{n}$ and ~$%
\vec{t}$ may not be linked to the boundary.

The components of the velocity in the new system will be denoted by $U_{\xi }$ and 
$U_{\zeta }$. We have:

\begin{equation*}
\left( 
\begin{array}{c}
U_{\xi } \smallskip\\ 
U_{\zeta }%
\end{array}
\right) =\left( 
\begin{array}{cc}
\dfrac{\partial \xi }{\partial x} & \dfrac{\partial \xi }{\partial y} \smallskip\\ 
\dfrac{\partial \zeta }{\partial x} & \dfrac{\partial \zeta }{\partial y}%
\end{array}
\right) \left( 
\begin{array}{c}
U \smallskip\\ 
V%
\end{array}
\right) =\left( 
\begin{array}{c}
U\dfrac{\partial \xi }{\partial x}+V\dfrac{\partial \xi }{\partial y} \smallskip\\ 
-U\dfrac{\partial \xi }{\partial y}+V\dfrac{\partial \xi }{\partial x}%
\end{array}
\right)
\end{equation*}

and:

\begin{equation*}
\left( 
\begin{array}{c}
U \smallskip\\ 
V%
\end{array}
\right) =\left( 
\begin{array}{cc}
\dfrac{\partial \xi }{\partial x} & -\dfrac{\partial \xi }{\partial y} \smallskip\\ 
\dfrac{\partial \xi }{\partial y} & \dfrac{\partial \xi }{\partial x}%
\end{array}
\right) \left( 
\begin{array}{c}
U_{\xi } \smallskip\\ 
U_{\zeta }%
\end{array}
\right) =\left( 
\begin{array}{c}
U_{\xi }\dfrac{\partial \xi }{\partial x}-U_{\zeta }\dfrac{\partial \xi }{%
\partial y} \smallskip\\ 
U_{\xi }\dfrac{\partial \xi }{\partial y}+U_{\zeta }\dfrac{\partial \xi }{%
\partial x}%
\end{array}
\right)
\end{equation*}

We first want to put the system in the form:

\begin{equation}
\dfrac{\partial F}{\partial t}+A_{x}\dfrac{\partial F}{\partial x}+B_{y}\dfrac{%
\partial F}{\partial x}=S(F)  \label{svtsyst}
\end{equation}

where $A_{x}$ and $B_{y}$ are matrices. For this goal:

In Equation \ref{svt}:

$\dfrac{\partial }{\partial x}(g\dfrac{h^{2}}{2})$ is written $c^{2}\dfrac{%
\partial h}{\partial x}$

$\dfrac{\partial }{\partial x}(hUU)$ is written $U^{2}\dfrac{\partial h}{%
\partial x}+h\dfrac{\partial U^{2}}{\partial x}=U^{2}\dfrac{\partial h}{%
\partial x}+2Uh\dfrac{\partial U}{\partial x}=U^{2}\dfrac{\partial h}{\partial
x}+2U\dfrac{\partial (hU)}{\partial x}-2U^{2}\dfrac{\partial h}{\partial x}$

$\dfrac{\partial }{\partial y}(hUV)$ is written $V\dfrac{\partial }{\partial y}%
(hU)+hU\dfrac{\partial V}{\partial y}=V\dfrac{\partial }{\partial y}(hU)+U%
\dfrac{\partial (hV)}{\partial y}-UV\dfrac{\partial h}{\partial y}$

$\dfrac{\partial }{\partial y}(g\dfrac{h^{2}}{2})$ is written $c^{2}\dfrac{%
\partial h}{\partial y}$

$\dfrac{\partial }{\partial x}(hVV)$ is written $-V^{2}\dfrac{\partial h}{%
\partial y}+2V\dfrac{\partial (hV)}{\partial y}$

$\dfrac{\partial }{\partial x}(hUV)$ is written $Vd\dfrac{\partial }{\partial x}%
(hU)+U\dfrac{\partial (hV)}{\partial x}-UV\dfrac{\partial h}{\partial x}$

We effectively get to Equation \ref{svtsyst} with:

\begin{equation}
A_{x}=\left( 
\begin{array}{ccc}
0 & 1 & 0 \smallskip\\ 
c^{2}-U^{2} & 2U & 0 \smallskip\\ 
UV & V & U%
\end{array}
\right) \text{ and }B_{y}=\left( 
\begin{array}{ccc}
0 & 0 & 1 \smallskip\\ 
UV & V & U \smallskip\\ 
c^{2}-V^{2} & 0 & 2V%
\end{array}
\right)
\end{equation}

Now we change the coordinates by writing that for every function $f$ we have:

\begin{equation*}
\dfrac{\partial f}{\partial x}=\dfrac{\partial \xi }{\partial x}\dfrac{%
\partial f}{\partial \xi }+\dfrac{\partial \zeta }{\partial x}\dfrac{\partial
f}{\partial \zeta }\text{ and }\dfrac{\partial f}{\partial y}=\dfrac{\partial
\xi }{\partial y}\dfrac{\partial f}{\partial \xi }+\dfrac{\partial \zeta }{%
\partial y}\dfrac{\partial f}{\partial \zeta }
\end{equation*}

It gives us a system in the form:

\begin{equation}
\dfrac{\partial F}{\partial t}+A_{\xi }\dfrac{\partial F}{\partial \xi }%
+B_{\zeta }\dfrac{\partial F}{\partial \zeta }=S(F)
\end{equation}
with:

\begin{equation*}
A_{\xi }=\dfrac{\partial \xi }{\partial x}A_{x}+\dfrac{\partial \xi }{%
\partial y}B_{y}\text{ and }B_{\zeta }=\dfrac{\partial \zeta }{\partial x}%
A_{x}+\dfrac{\partial \zeta }{\partial y}B_{y}=-\dfrac{\partial \xi }{%
\partial y}A_{x}+\dfrac{\partial \xi }{\partial x}B_{y}
\end{equation*}
which gives:

\begin{equation}
A_{\xi }=\left( 
\begin{array}{ccc}
0 & \dfrac{\partial \xi }{\partial x} & \dfrac{\partial \xi }{\partial y} \smallskip\\ 
\dfrac{\partial \xi }{\partial x}(c^{2}-U^{2})-\dfrac{\partial \xi }{%
\partial y}UV & 2U\dfrac{\partial \xi }{\partial x}+V\dfrac{\partial \xi }{%
\partial y} & U\dfrac{\partial \xi }{\partial y}\smallskip \\ 
\dfrac{\partial \xi }{\partial y}(c^{2}-V^{2})-\dfrac{\partial \xi }{%
\partial x}UV & V\dfrac{\partial \xi }{\partial x} & U\dfrac{\partial \xi }{%
\partial x}+2V\dfrac{\partial \xi }{\partial y}%
\end{array}
\right) \text{ }
\end{equation}
and:

\begin{equation}
B_{_{\zeta }}=\left( 
\begin{array}{ccc}
0 & -\dfrac{\partial \xi }{\partial y} & \dfrac{\partial \xi }{\partial x}
\smallskip\\ 
-\dfrac{\partial \xi }{\partial y}(c^{2}-u^{2})-\dfrac{\partial \xi }{%
\partial x}UV & -2U\dfrac{\partial \xi }{\partial y}+V\dfrac{\partial \xi }{%
\partial x} & U\dfrac{\partial \xi }{\partial x} \smallskip\\ 
\dfrac{\partial \xi }{\partial x}(c^{2}-V^{2})+\dfrac{\partial \xi }{%
\partial y}UV & -V\dfrac{\partial \xi }{\partial y} & -U\dfrac{\partial \xi 
}{\partial y}+2V\dfrac{\partial \xi }{\partial x}%
\end{array}
\right)
\end{equation}
or even, still denoting $U_{\xi }$ as the normal component of velocity and $%
U_{\zeta }$ the tangential component:

\begin{equation}
A_{\xi }=\left( 
\begin{array}{ccc}
0 & \dfrac{\partial \xi }{\partial x} & \dfrac{\partial \xi }{\partial y} \\ 
\dfrac{\partial \xi }{\partial x}c^{2}-UU_{\xi } & U\dfrac{\partial \xi }{%
\partial x}+U_{\xi } & U\dfrac{\partial \xi }{\partial y} \\ 
\dfrac{\partial \xi }{\partial y}c^{2}-VU_{\xi } & V\dfrac{\partial \xi }{%
\partial x} & U_{\xi }+V\dfrac{\partial \xi }{\partial y}%
\end{array}
\right) \text{ }
\end{equation}
and:

\begin{equation}
\text{ }B_{_{\zeta }}=\left( 
\begin{array}{ccc}
0 & -\dfrac{\partial \xi }{\partial y} & \dfrac{\partial \xi }{\partial x}
\\ 
-\dfrac{\partial \xi }{\partial y}c^{2}-UU_{\zeta } & -U\dfrac{\partial \xi 
}{\partial y}+U_{\zeta } & U\dfrac{\partial \xi }{\partial x} \\ 
\dfrac{\partial \xi }{\partial x}c^{2}-UU_{\zeta } & -V\dfrac{\partial \xi }{%
\partial y} & U_{\zeta }+V\dfrac{\partial \xi }{\partial x}%
\end{array}
\right)
\end{equation}

Subsequently, we ignore the variations along the direction $\zeta $ and try
to solve the system:

\begin{equation}
\dfrac{\partial F}{\partial t}+A_{\xi }\dfrac{\partial F}{\partial \xi }=S(F)
\end{equation}
An open question is: which part of $S(F)$ should be kept in this equation ?
We discard $Sce$, $F_{x}$ and $F_{y}$, and keep only the variations of
bottom along the direction $\xi $. It gives

\begin{equation}
S_{\xi }(F)=\left( 
\begin{array}{c}
0 \\ 
-gh\dfrac{\partial \xi }{\partial x}\dfrac{\partial b}{\partial \xi } \\ 
-gh\dfrac{\partial \xi }{\partial y}\dfrac{\partial b}{\partial \xi }%
\end{array}
\right)
\end{equation}

For the time being, we call it $S_{\xi }(F)$ whatever its value and go on
with the diagonalisation of $A_{\xi }$.

Now $A_{\xi }$ is diagonalized as $A_{\xi }=L^{-1}\Lambda L$ with:

\begin{equation*}
L=\left( 
\begin{array}{ccc}
U_{\zeta } & \dfrac{\partial \xi }{\partial y} & -\dfrac{\partial \xi }{%
\partial x} \\ 
c-U_{\xi } & \dfrac{\partial \xi }{\partial x} & \dfrac{\partial \xi }{%
\partial y} \\ 
c+U_{\xi } & -\dfrac{\partial \xi }{\partial x} & -\dfrac{\partial \xi }{%
\partial y}%
\end{array}
\right)
\end{equation*}
and:

\begin{equation*}
\Lambda =\left( 
\begin{array}{ccc}
U_{\xi } & 0 & 0 \\ 
0 & U_{\xi }+c & 0 \\ 
0 & 0 & U_{\xi }-c%
\end{array}
\right)
\end{equation*}

This can be controlled by checking that $LA_{\xi }=\Lambda L$.

By stating that $dW=LdF$, we then get back to the diagonalized system:

\begin{equation}
\dfrac{\partial W}{\partial t}+\Lambda \dfrac{\partial W}{\partial \xi }%
=LS_{\xi }  \label{equacaract}
\end{equation}
each of whose lines is a simple transport equation with source term.
Thompson proposes to consider that $L$ is constant in the vicinity of a
boundary point, and to write $W=\overline{L}F$, where:

\begin{equation}
\overline{L}=\left( 
\begin{array}{ccc}
\overline{U_{\zeta }} & \dfrac{\partial \xi }{\partial y} & -\dfrac{\partial
\xi }{\partial x} \\ 
\overline{c}-\overline{U_{\xi }} & \dfrac{\partial \xi }{\partial x} & 
\dfrac{\partial \xi }{\partial y} \\ 
\overline{c}+\overline{U_{\xi }} & -\dfrac{\partial \xi }{\partial x} & -%
\dfrac{\partial \xi }{\partial y}%
\end{array}
\right)
\end{equation}
the overbar values being considered as constant (these are the values
deduced from the local conditions: h, U and V at the original starting point
of the characteristics). The Riemann invariants of the vector $W$ are thus:

\begin{itemize}
\item $W_{1}=h(\overline{U_{\zeta }}-U_{\zeta })$ \ \ \ (advection with
velocity $U_{\xi }$).

\item $W_{2}=h(\overline{c}+U_{\xi }-\overline{U_{\xi }})$ \ \ \
(advection with velocity $U_{\xi }+c$).

\item $W_{3}=h(\overline{c}-U_{\xi }+\overline{U_{\xi }})$ \ \ \
(advection with velocity $U_{\xi }-c$).
\end{itemize}

\noindent and to which can be added, if a tracer $T$ also has to be
considered:

\begin{itemize}
\item $W_{4}=h(T-\overline{T})$ \ \ \ (advection with velocity $U_{\xi }$).
\end{itemize}

Pure advection is treated with the method of characteristics. To be more
precise, a first advection is done with velocity $U_{\xi }.\ $This is done
backwards in time. For every boundary point of Thompson type, we compute the
backward trajectory and find, at what is called the foot of the
characteristic curve (starting point of the trajectory which will arrive at
the boundary point after $\Delta t$), the values of depth and components of
velocity which we call $\widetilde{h}_{1}$, $\widetilde{U}_{1}$, $\widetilde{%
V}_{1}$ and $\widetilde{T}_{1}$. If we neglect the source terms and take the
invariants at this foot of characteristic pathline, we have:

\begin{itemize}
\item $W_{1}=h(\overline{U_{\zeta }}-U_{\zeta })=\widetilde{W}_{1}=%
\widetilde{h}_{1}(\overline{U_{\zeta }}-\widetilde{U}_{\zeta 1})$ with $%
\overline{U_{\zeta }}=-U\dfrac{\partial \xi }{\partial y}+V\dfrac{\partial
\xi }{\partial x}$ and $\widetilde{U}_{\zeta 1}=-\widetilde{U}\dfrac{%
\partial \xi }{\partial y}+\widetilde{V}\dfrac{\partial \xi }{\partial x}$

\item $W_{4}=h(T-\overline{T})=\widetilde{h}(\widetilde{T}_{1}-\overline{T}) 
$ with $\overline{T}=T$
\end{itemize}

then, after an advection with velocity $U_{\xi }+c$ , i.e.\ with results
now called $\widetilde{h}_{2}$, $\widetilde{U}_{2}$ and $\widetilde{V}_{2}$:

\begin{itemize}
\item $W_{2}=h(\overline{c}+U_{\xi }-\overline{U_{\xi }})=\widetilde{W}%
_{2}=\widetilde{h}_{2}(\overline{c}+\widetilde{U}_{\xi 2}-\overline{U_{\xi
}})$ with $\overline{c}=\sqrt{gh}$, $\overline{U_{\xi }}=U\dfrac{\partial
\xi }{\partial x}+V\dfrac{\partial \xi }{\partial y}$ and $\widetilde{U}%
_{\xi 2}=\widetilde{U}_{2}\dfrac{\partial \xi }{\partial x}+\widetilde{V}%
_{2}\dfrac{\partial \xi }{\partial y}$.
\end{itemize}

then, after an advection with velocity $U_{\xi }-c$, i.e.\ with yet other
values denoted $\widetilde{h}_{3}$, $\widetilde{U}_{3}$ and $\widetilde{V}%
_{3}$:

\begin{itemize}
\item $W_{3}=h(\overline{c}-U_{\xi }+\overline{U_{\xi }})$ $=$\ $%
\widetilde{W}_{3}=$\ $\widetilde{h}_{3}(\overline{c}-\widetilde{U}_{\xi 3}+%
\overline{U_{\xi }})$\ with $\overline{c}=\sqrt{gh}$, $\overline{U_{\xi }}%
=U\dfrac{\partial \xi }{\partial x}+V\dfrac{\partial \xi }{\partial y}$ and $%
\widetilde{U}_{\xi 3}=\widetilde{U}_{3}\dfrac{\partial \xi }{\partial x}+%
\widetilde{V}_{3}\dfrac{\partial \xi }{\partial y}$.
\end{itemize}

All this is valid only if the backwards characteristic goes inside the
domain.\ This can be checked by the fact that $\vec{U}_{conv}.%
\vec{n}>0$, where $\vec{U}_{conv}$\ is the advection
velocity field (i.e.\ based on $U_{\xi }$, $U_{\xi }+c$ or $U_{\xi }-c$,
respectively for $W_{1}$, $W_{2}$ and $W_{3}$). If $\vec{U}%
_{conv}.\vec{n}<0$, all variables with a tilde will be based on
the boundary conditions prescribed by the user. For example, $\widetilde{U}%
_{\zeta 1}$ may be taken equal to $-U_{bor}\dfrac{\partial \xi }{\partial y}%
+V_{bor}\dfrac{\partial \xi }{\partial x}$, where $U_{bor}$ and $V_{bor}$
are the prescribed components of the velocity field.

Source terms will be considered later. Once the Riemann invariants are
known, the primitive variables can be restored by the following formulae:

\begin{equation}
h=\dfrac{W_{2}+W_{3}}{2\overline{c}}  \label{equ1}
\end{equation}

\begin{equation}
h(U-\overline{U})=\dfrac{\partial \xi }{\partial y}W_{1}+\dfrac{\partial \xi 
}{\partial x}(W_{2}-W_{3})  \label{equ2}
\end{equation}

\begin{equation}
h(V-\overline{V})=\dfrac{\partial \xi }{\partial y}(W_{2}-W_{3})-\dfrac{%
\partial \xi }{\partial x}W_{1}  \label{equ3}
\end{equation}

\begin{equation}
h(T-\overline{T})=-W_{4}  \label{equ4}
\end{equation}

Equation \ref{equ1}\ can be used to eliminate h from the 3 others, it yields:

\begin{equation}
h=\dfrac{W_{2}+W_{3}}{2\overline{c}}
\end{equation}

\begin{equation}
hU=\dfrac{W_{2}+W_{3}}{2\overline{c}}\overline{U}+\dfrac{\partial \xi }{%
\partial y}W_{1}+\dfrac{\partial \xi }{\partial x}(W_{2}-W_{3})
\end{equation}

\begin{equation}
hV=\dfrac{W_{2}+W_{3}}{2\overline{c}}\overline{V}+\dfrac{\partial \xi }{%
\partial y}(W_{2}-W_{3})-\dfrac{\partial \xi }{\partial x}W_{1}
\end{equation}

\begin{equation}
hT=\dfrac{W_{2}+W_{3}}{2\overline{c}}\overline{T}-W_{4}
\end{equation}

This form is not the most practical but readily gives, if necessary or for
checking:

\begin{equation}
\overline{L}^{-1}=\left( 
\begin{array}{ccc}
0 & \dfrac{1}{2\overline{c}} & \dfrac{1}{2\overline{c}} \\ 
\dfrac{\partial \xi }{\partial y} & \dfrac{1}{2}\dfrac{\partial \xi }{%
\partial x}+\dfrac{\overline{U}}{2\overline{c}} & -\dfrac{1}{2}\dfrac{\partial
\xi }{\partial x}+\dfrac{\overline{U}}{2\overline{c}} \\ 
-\dfrac{\partial \xi }{\partial x} & \dfrac{1}{2}\dfrac{\partial \xi }{%
\partial y}+\dfrac{\overline{V}}{2\overline{c}} & -\dfrac{1}{2}\dfrac{\partial
\xi }{\partial y}+\dfrac{\overline{V}}{2\overline{c}}%
\end{array}
\right)
\end{equation}

We will favour the following formulas for the implementation:

\begin{equation}
h=\dfrac{W_{2}+W_{3}}{2\overline{c}}
\end{equation}

\begin{equation}
U=\dfrac{\dfrac{\partial \xi }{\partial y}W_{1}+\dfrac{\partial \xi }{%
\partial x}(W_{2}-W_{3})}{h}+\overline{U}
\end{equation}

\begin{equation}
V=\dfrac{-\dfrac{\partial \xi }{\partial x}W_{1}+\dfrac{\partial \xi }{%
\partial y}(W_{2}-W_{3})}{h}+\overline{V}
\end{equation}

\begin{equation}
T=-\dfrac{W_{4}}{h}+\overline{T}
\end{equation}

If we do not neglect source terms, they have to be integrated along the
characteristic curve. Assuming a constant $\overline{L}$ as done before we
have:

\begin{equation}
\left( 
\begin{array}{c}
W_{1} \\ 
W_{2} \\ 
W_{3}%
\end{array}
\right) =\left( 
\begin{array}{c}
\widetilde{W}_{1} \\ 
\widetilde{W}_{2} \\ 
\widetilde{W}_{3}%
\end{array}
\right) +\Delta t\left( 
\begin{array}{c}
\overline{U_{\zeta }}~Sce+\dfrac{\partial \xi }{\partial y}(-gh\dfrac{%
\partial \xi }{\partial x}\dfrac{\partial b}{\partial \xi }+hF_{x})-%
\dfrac{\partial \xi }{\partial x}(-gh\dfrac{\partial \xi }{\partial y}\dfrac{%
\partial b}{\partial \xi }+hF_{y}) \\ 
(\overline{c}-\overline{U_{\xi }})Sce+\dfrac{\partial \xi }{\partial x}(-gh%
\dfrac{\partial \xi }{\partial x}\dfrac{\partial b}{\partial \xi }%
+hF_{x})+\dfrac{\partial \xi }{\partial y}(-gh\dfrac{\partial \xi }{\partial
y}\dfrac{\partial b}{\partial \xi }+hF_{y}) \\ 
(c+U_{\xi })Sce-\dfrac{\partial \xi }{\partial x}(-gh\dfrac{\partial \xi }{%
\partial x}\dfrac{\partial b}{\partial \xi }+hF_{x})-\dfrac{\partial \xi 
}{\partial y}(-gh\dfrac{\partial \xi }{\partial y}\dfrac{\partial b}{%
\partial \xi }+hF_{y})%
\end{array}
\right)
\end{equation}

Neglecting again $Sce$, $F_{x}$ and $F_{y}$, we are left with:

\begin{equation*}
\left( 
\begin{array}{c}
W_{1} \\ 
W_{2} \\ 
W_{3}%
\end{array}
\right) =\left( 
\begin{array}{c}
\widetilde{W}_{1} \\ 
\widetilde{W}_{2} \\ 
\widetilde{W}_{3}%
\end{array}
\right) -gh\Delta t\left( 
\begin{array}{c}
0 \\ 
\dfrac{\partial b}{\partial \xi } \\ 
-\dfrac{\partial b}{\partial \xi }%
\end{array}
\right)
\end{equation*}

Though the source terms could be treated in an explicit way, we do the
following approximation: $\dfrac{\partial b}{\partial \xi }$ is
approximated as $\dfrac{b-\widetilde{Z}_{f}}{(U+c)\Delta t}$, i.e. the
variation of $b$ along the (backwards) characteristic curve divided by
the length of the curve, then $U$ is neglected so that we have $\dfrac{%
\partial b}{\partial \xi }\simeq \dfrac{b-\widetilde{Z}_{f}}{c\Delta
t}$, and eventually $-gh\Delta t\dfrac{\partial b}{\partial \xi }\ $is
simplified into $-c(\widetilde{Z}_{f}-b)$. It gives the following new
formulas for $W_{2}$ and $W_{3}$: 
\begin{equation*}
W_{2}=\overline{c}(\widetilde{h}_{2}+\widetilde{Z}_{f2}-b)+\widetilde{h}%
_{2}(\widetilde{U}_{\xi 2}-\overline{U_{\xi }})
\end{equation*}
\begin{equation*}
W_{3}=\overline{c}(\ \widetilde{h}_{3}+\widetilde{Z}_{f3}-b)+\ 
\widetilde{h}_{3}(-\widetilde{U}_{\xi 3}+\overline{U_{\xi }})
\end{equation*}


In Thompson publication finite differences are employed to solve the 3
advection problems of the method, this was done mainly because at that time
regular grids were common practice. Eric David, at Sogreah, then resorted to
the method of characteristics itself to solve these problems on unstructured
grids. At that time (1999) it precluded parallelism. Then Jacek Jankowski
(BAW Karlsruhe) wrote an amazing parallel version of the method of
characteristics (module \textquotedblright streamline\textquotedblright\ in
library BIEF). More recently, module streamline was adapted by Christophe
Denis (Sinetics, EDF\ R\&D) for dealing with a list of points that are not
necessarily linked to mesh nodes, to enable the treatment of particles on
one hand, and Thompson boundary points on the other hand. This was not the
end of the story.\ As a matter of fact, the advection fields requested by
Thompson boundary points depend on the starting point, and these specific
fields must be defined for the whole domain.\ In parallel this implies that
every Thompson boundary point has to send its advection fields to all
processors, in case its characteristic pathlines would go to another
sub-domain. This was considered too cumbersome, a dead end.\ Moreover, the
Thompson theory leads to the fact that two nearby boundary points may have
their characteristics pathlines crossing, because linearisation was done in
two different directions.\ This is somewhat against the nature of
characteristics that do not cross unless they carry the same invariant. For
all these reasons it was considered that the theory had to be modified.\ It
seems natural that the linearisation direction should be the direction of
the flow. It is what is attempted here. \smallskip
%We shall first fully explain what
%was done in previous versions, and then we shall move to the new idea. \smallskip

\textit{The new theory:} \\ \smallskip

Linearisation in the direction of the flow: \\

All what has been said in previous section is valid up to version 6.0 if we
choose for $\vec{n}$ the outward normal vector to the boundary.\
The problem is that in this case the 3 advections fields depend on the
boundary point under treatment.\ This was heavy in scalar mode, where points
with the same normal were grouped for optimization and shared the same
advection field. It becomes even more heavy in parallel because these
advection fields should be built for the whole domain, which implies that
for every Thompson point, its normal vector must be exported to all
sub-domains. It also appears very strange that characteristics of the same
family stemming from two different points may cross because they have a
different original direction.

The new theory consists in choosing advection fields that would not depend
on a given boundary point.\ It seems very natural to choose, instead of the
outward normal vector $\vec{n}$, the direction of the velocity
field itself. We have then:

\begin{equation}
\vec{n}=\left( 
\begin{array}{c}
\dfrac{\partial \xi }{\partial x} \\ 
\dfrac{\partial \xi }{\partial y}%
\end{array}
\right) =\left( 
\begin{array}{c}
\dfrac{U}{\sqrt{U^{2}+V^{2}}} \\ 
\dfrac{V}{\sqrt{U^{2}+V^{2}}}%
\end{array}
\right)  \label{newtheory}
\end{equation}

An important consequence of this choice is that the velocity $\overline{U}%
_{\zeta }$ is always 0 by definition, which would lead to $W_{1}=0$. This is
true in fact only if we consider that the direction $\vec{n}$
changes along characteristics, it is false if we keep the original $%
\vec{n}$, which would be consistent with the linearisation
leading to $\overline{L}$. Tests show that it is better to consider that $%
U_{\zeta }$ is indeed not 0, thus sticking to the linearisation. A
possibility that remains to be tested would be considering that $U_{\zeta }$
is indeed 0, and taking the norm of velocity for the component $U_{\xi }$.

In any case there is an obvious problem when there is no velocity, the
direction where to apply the celerity $c$ is then undefined. A first idea is
to cancel also the celerity $c$ in this case, so that all variables will
keep their original value. This is not possible, because a velocity equal to
0 for a given boundary point would then trigger that the depth and velocity
at this point remain unchanged.\ This is valid only if there is no wave
approaching the point, i.e. no velocity and no free surface slope. When
there is a free surface slope, it seems then natural to choose the direction
of the vector $-g\,\vec{grad}(\eta)$, which is the driving term
in momentum equation that will create velocity at the next time step. This
happens to be very important in tests, especially the gaussian hill test
case.

Depth and velocity fields for interpolation:\\

An unexpected problem occured in the results, showing that the tests $%
\vec{U}_{conv}.\vec{n}>0$, to decide whether we should
take e.g. the depth $h$ or the prescribed depth $h_{bor}$ for computing $%
\widetilde{h}$, could happen to be wrong.\ As a matter of fact the method of
characteristics itself is able to check if the pathline goes out of the
domain, and in this case it stops and interpolates at this exit point. In a
corner the average $\vec{n}$ of the corner point may lead to a
different decision, thus leading to wrongly choose for example $h$ instead
of $h_{bor}$. Any case where the value $\vec{U}_{conv}.%
\vec{n}$ is very close to 0 will lead to a random choice, and
then to large differences if $h$ and $h_{bor}$\ are very different. It was
thus decided to discard the tests $\vec{U}_{conv}.\vec{%
n}>0$ and to use interpolation fields of $h$, $U$ and $V$ that already
contain the prescribed boundary conditions. A characteristic pathline that
exits a Thompson boundary will thus find naturally that $\widetilde{h}%
=h_{bor}$, without resorting to testing $\vec{U}_{conv}.%
\vec{n}$.\ A\ drawback is that for small Courant numbers, when
the characteristics pathlines will not go far from boundaries, their
interpolated values will be influenced by the prescribed values of the
boundary.\ When prescribed values are correct, which is generally the case
with box models and measurements, this could be also an advantage. With this
new approach there can be no discontinuity of choice due to a truncation
error.