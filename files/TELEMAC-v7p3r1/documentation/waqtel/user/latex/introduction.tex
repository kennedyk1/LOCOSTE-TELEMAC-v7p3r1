\chapter{Introduction}

Waqtel (WAter Quality for TELemac) is a component of the Telemac-Mascaret system (TMS) which focuses on the water quality 
aspects. It was developped to allow the TMS's users to tackle water quality problems together with hydrodynamics.\newline
Up to release V7P0, \telemac{2D} and \telemac{3D} were coupled with DELWAQ, the Deltares water quality code. This coupling,
though working well for simple and medium sized models, was not suitable and cumbersome for big models. The main issue 
related to the use of Telemac-DELWAQ was the uncompatible parallelization of both codes. \newline
To overcome this issue and in order to fully benefit from the parallelization effciency of the TMS, the development team
introduced a first version of \waqtel in the V7P1 release. 

Waqtel is developed by the LNHE (Laboratoire National d'Hydraulique et
Environnement) of the Research and Development Division of EDF (EDF-R\&D). As
for previous versions, the 7.1 release of the code complies with the Quality
Assurance procedures of scientific and technical softwares of EDF-R\&D. It is a
process of construction and verification of the product quality in the
different phases of his life. In particular, a software following the Quality
Assurance procedures comes with a Validation Folder that describes the intended
use of the software and a set of test cases. This document allows you to judge
the performance and limitations of the software, situating the field of
application. These tests are also used in the development of the software and are
checked at every new release.

\section{Position of the \telemac{3d} code within the telemac modelling system}

The \waqtel software is part of the TELEMAC modelling system developed by
the LNHE of EDF R\&D. TELEMAC is a set of modelling tools allowing to treat
every aspects of natural free surface hydraulics: currents, waves, transport of
tracers and sedimentology.
\newline
\waqtel, unlike other compnents of the TMS, can not be run in a stand-alone mode.
To run a \waqtel model, it is necessary to run \telemac{2D} or \telemac{3D} coupled 
with \waqtel using the keyword COUPLING WITH= 'WAQTEL' (in french: COUPLAGE AVEC ='WAQTEL')

The pre-processing and post-processing of simulations can be done either
directly within the TELEMAC system or with different software that present an
interface of communication with the system. We can particularly mention the
following tools:

\begin{itemize}
\item The FUDAA-PREPRO software, developed from the FUDAA platform
by the CEREMA's Recherche, Informatique et Modélisation Department, covers all
the pre-processing tasks involved by the achievement of a numerical hydraulic
study, as well as a graphical post-processing tool,
\item The Blue Kenue software, developed the Hydraulic Canadian
Center, proposes a powerful mesh generation tool and a user-friendly
post-processing tool,
\item The Janet software, developed by Smile Consult GmbH, which offers among
others, a mesh generation tool,
\item The ParaView software, developed by Sandia National Laboratories, Los
Alamos National Laboratory and Kitware, which enables to visualise 3D results,
big data in particular and is open source,
\item The SALOME-HYDRO software based on the SALOME platform, developed by EDF,
CEA and OPENCASCADE which enables to handle raw data (bathymetry, maps,
pictures, LIDAR\ldots) until the mesh generation.
The post-processing tool ParaViS available in the SALOME platform is based on
the ParaView software and can visualise 1D, 2D or 3D results.
A first version of SALOME-HYDRO has been available since Spring 2016,
\item The QGIS software, which is an open source Geographic Information System.
\end{itemize}


\section{Software environment}

All the simulation modules are written in FORTRAN 90, with no use of the
specific language extensions in a given machine. They can be run on all the PCs
(or PC "clusters") under Windows and Linux operating systems as well as on the
workstations under the Unix operating system.

\section{User programming}

When using a simulation module from the TELEMAC system, the user may have to
program specific subroutines which are not in the code's standard release. In
particular, that is made through a number of so-called «~user~» subroutines.
These subroutines are written so that they can be modified, provided that the
user has a basic knowledge in FORTRAN language, with the help of the «~Guide
for programming in the Telemac system~» \cite{HervouetProg2009}.

The procedure to be carried out in that case comprises the steps of:

\begin{itemize}
\item Recovering the standard version of the user subroutine(s) as supplied in
the distribution and copying it into the current directory,
\item Amending the subroutine(s) according to the model to be constructed,
\item Concatenating the whole set of subroutines into a single FORTRAN file
which will be compiled during the \telemac{2D} or \telemac{3D} launching process.
\end{itemize}

During that programming stage, the user can gain access to the various
variables of the software through the FORTRAN 90 structures.

All the data structures are gathered within FORTRAN files, which are known as
modules. For \waqtel, the file name is \telfile{DECLARATION\_WAQTEL.f}. To gain
access to the \waqtel data, just insert the command \telfile{USE
DECLARATIONS\_WAQTEL} into the beginning of the subroutine. Adding the
command \telfile{USE BIEF} may also be necessary in order to reach the structure in the
\bief library.

Nearly all the arrays which are used by \waqtel are declared in the form of
a structure. For example, the access to the water depth array will be in the
form \telfile{H\%R}, \telfile{\%R} meaning it is a real-typed pointer.
In case of an integer-typed pointer, the \telfile{\%R} is replaced by a \telfile{\%I}.
However, in order to avoid having to handle too many \telfile{\%R} and \telfile{\%I},
a number of aliases are defined, such as, the \telfile{NPOIN3}, \telfile{NELEM3} and
\telfile{NPTFR2} variables. For further details, the user can refer to
the programming guide in TELEMAC \cite{HervouetProg2009}.
