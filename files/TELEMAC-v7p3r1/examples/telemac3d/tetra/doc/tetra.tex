% case name
\chapter{tetra}
%
% - Purpose & Description:
%     These first two parts give reader short details about the test case,
%     the physical phenomena involved, the geometry and specify how the numerical solution will be validated
%
\section{Purpose}
%
This test demonstrates the ability of \telemac{3d} to be discretised
with prisms split into tetrahedra.\\
To compare the solution produced by \telemac{3d} in a frictionless
channel presenting an idealised bump on the bottom with the analytical
solution to this problem.\\
The flow regime is sub-critical.\\
The channel is horizontal with a 4~m long bump in its middle.
The maximum elevation of the bump is 20~cm.\\
The tracer used is salinity.
%
\section{Description}
%

%
% - Reference:
%     This part gives the reference solution we are comparing to and
%     explicits the analytical solution when available;
%
% bibliography can be here or at the end
%\subsection{Reference}
%
%
\subsection{Reference}
%

%
% - Geometry and Mesh:
%     This part describes the mesh used in the computation
%
%
\subsection{Geometry and Mesh}
%
\subsubsection{Bathymetry}
%
If 6.51~m < $x$ < 13.49~m, $z_{\rm{f}} = -0.0246875(x-10)^2$\\
If $x$ < 1.44~m, $z_{\rm{f}} = -0.2-0.3(\frac{x}{2.5})^2$\\
$z_{\rm{f}}$ = -0.3~m elsewhere
%
\subsubsection{Geometry}
%
It is the same geometry as the \telemac{2d} test case ``bumflu''.
%
\subsubsection{Mesh}
%
2,620 triangular elements\\
1,452 nodes\\
10 planes regularly spaced in the vertical direction except one fixed
plane with the elevation -0.2~m (plane number 4).
%
% - Physical parameters:
%     This part specifies the physical parameters
%
%
\subsection{Physical parameters}
%
Turbulence model: $k-\epsilon$ model\\
Coriolis: no
%
% Experimental results (if needed)
%\subsection{Experimental results}
%
% bibliography can be here or at the end
%\subsection{Reference}
%
% Section for computational options
%\section{Computational options}
%
% - Initial and boundary conditions:
%     This part details both initial and boundary conditions used to simulate the case
%
%
\subsection{Initial and Boundary Conditions}
%
\subsubsection{Initial conditions}
%
Constant initial water level at $z$ = 1.8~m\\
No velocity\\
Initial tracer: uniform 30
%
\subsubsection{Boundary conditions}
%
Upstream: imposed flow rate (4~m$^3$/s) and imposed tracer
(40 if $z \leq$ -0.2~m, 30 if $z$ > -0.2~m)\\
Downstream: prescribed elevation 1.8~m so that the water depth is 2~m
%
\subsection{General parameters}
%
Time step: 0.04~s\\
Simulation duration: 40~s
%
% - Numerical parameters:
%     This part is used to specify the numerical parameters used
%     (adaptive time step, mass-lumping when necessary...)
%
%
\subsection{Numerical parameters}
%
Non-hydrostatic version\\
Advection for velocities: edge by edge explicit finite volume Leo
Postma\\
Advection for tracers: edge by edge explicit finite volume Leo
Postma\\
Advection for $k-\epsilon$: edge by edge explicit finite volume Leo
Postma
%
\subsection{Comments}
%
% - Results:
%     We comment in this part the numerical results against the reference ones,
%     giving understanding keys and making assumptions when necessary.
%
%
\section{Results}
%

%
\section{Conclusion}
%

%
% Here is an example of how to include the graph generated by validateTELEMAC.py
% They should be in test_case/img
%\begin{figure} [!h]
%\centering
%\includegraphics[scale=0.3]{../img/mygraph.png}
% \caption{mycaption}\label{mylabel}
%\end{figure}
%
% bibliography
%\section{Reference}
