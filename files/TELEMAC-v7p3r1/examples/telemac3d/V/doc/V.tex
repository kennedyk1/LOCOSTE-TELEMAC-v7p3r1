% case name
\chapter{V}
%
% - Purpose & Description:
%     These first two parts give reader short details about the test case,
%     the physical phenomena involved, the geometry and specify how the numerical solution will be validated
%
\section{Purpose}
%
The purpose of this test is to verify the validity of the diffusion step
and the proper treatment of the buoyancy terms.
Moreover, this test demonstrates the ability of \telemac{3d} to model
a vertical stratification induced by an active tracer distribution
on a non-horizontal topography.
A closed rectangular channel is initialised with such vertical tracer
distribution without motion.
This stratification is stable and the distribution of the tracer should
not evolve in time neither generate any flow.
The test case is treated with both prism and tetrahedron elements.
%
\section{Description}
%
The considered domain is a horizontal V-shaped channel.
The horizontal mesh is composed of triangular cells nearly homogenous in
size.
The test case is first solved using a vertical mesh of prismatic cells
fitting the topography, and in a second time using tetrahedron elements.
The horizontal and vertical meshes are presented on figure 3.2.1.
The active tracer is the temperature with a corresponding value of the
thermal expansion coefficient $\beta = 2.10^{-4}$ K$^{-1}$
(see the User's Manual).
%
% - Reference:
%     This part gives the reference solution we are comparing to and
%     explicits the analytical solution when available;
%
% bibliography can be here or at the end
%\subsection{Reference}
%
%
\subsection{Reference}
%

%
% - Geometry and Mesh:
%     This part describes the mesh used in the computation
%
%
\subsection{Geometry and Mesh}
%
\subsubsection{Bathymetry}
%
Bottom at -13~m in the centre of the channel\\
Bottom at 0~m on the side of the channel banks\\
Linearly interpolated depth between the centre and the banks\\
The bathymetry can be observed on figure 3.2.1
(with an expansion factor on the vertical equal to 10)
%
\subsubsection{Geometry}
%
Channel length = 500~m\\
Channel width = 100~m
%
\subsubsection{Mesh}
%
648 triangular elements\\
373 nodes\\
11 planes regularly spaced in the vertical direction\\
Prism or tetrahedron elements
%
% - Physical parameters:
%     This part specifies the physical parameters
%
%
\subsection{Physical parameters}
%
Constant horizontal viscosity 1~m$^2$/s, and tracer diffusion 1~m$^2$/s\\
Temperature density law specifying an expansion coefficient
$\beta = 2.10^{-4}$~K$^{-1}$\\
No vertical viscosity\\
Coriolis: no\\
Wind: no
%
% Experimental results (if needed)
%\subsection{Experimental results}
%
% bibliography can be here or at the end
%\subsection{Reference}
%
% Section for computational options
%\section{Computational options}
%
% - Initial and boundary conditions:
%     This part details both initial and boundary conditions used to simulate the case
%
%
\subsection{Initial and Boundary Conditions}
%
\subsubsection{Initial conditions}
%
No velocity\\
Constant initial water level at $z = 0.1$~m\\
Tracer initialised as $T = 10 + z/1.3~^\circ$C
%
\subsubsection{Boundary conditions}
%
Channel banks: slip solid boundary\\
Bottom: slip solid boundary\\
Tracer: no flux through the bottom and the free surface
%
\subsection{General parameters}
%
Time step: 0.1~s\\
Simulation duration: 1~s
%
% - Numerical parameters:
%     This part is used to specify the numerical parameters used
%     (adaptive time step, mass-lumping when necessary...)
%
%
\subsection{Numerical parameters}
%
Non-hydrostatic computation\\
Advection for velocities: Explicit Leo Postma scheme\\
Advection for tracer: Explicit Leo Postma scheme
%
\subsection{Comments}
%
% - Results:
%     We comment in this part the numerical results against the reference ones,
%     giving understanding keys and making assumptions when necessary.
%
%
\section{Results}
%
During the simulation, the vertical profile of tracer concentration is
stable for both computations (see figure 3.2.2 and 3.2.4).
However, at the end of the computation with prism elements,
some negligible differences on the temperature field (maximum value
for example, up to 0.01 \%) are observed.
Figure 3.2.3 shows disturbance of velocity field negligible
(of the order of 10$^{-7}$) for the prism elements computation.
This velocity field is connected to the difficulty to build the
diffusion matrix using prismatic elements.
The computation with tetrahedron elements (see figure 3.2.5)
shows some very negligible (of the order of 10$^{-16}$) disturbance
of the velocity field.
In any case, the situation can be considered as globally stable.
The test is done essentially to verify the horizontal diffusion terms,
which are estimated in the transformed $\sigma$-mesh (buoyancy terms).
In this case, the tracer does not induce any flow.
The final mass balance of computation using prism elements exposes a
very good mass conservation (the total mass loss is less than
0.5 10$^{-6}$).
For the computation using tetrahedron elements, the conservation of
mass and temperature is perfect.
%
\section{Conclusion}
%
The diffusion equation of tracers is properly solved by \telemac{3d}.
The buoyancy terms are properly taken into account for a linear vertical
 tracer distribution.
Unlike tetrahedron elements computation, using prism elements generates
a small disturbance of the velocity field and some negligible
differences on the temperature field.
Nevertheless, the generated disturbance is small with respect to the
vertical gradient of the tracer and is principally due to machine
precision and diffusion matrix treatment.
The tracer is still stable at the end of the simulation.
%
% Here is an example of how to include the graph generated by validateTELEMAC.py
% They should be in test_case/img
%\begin{figure} [!h]
%\centering
%\includegraphics[scale=0.3]{../img/mygraph.png}
% \caption{mycaption}\label{mylabel}
%\end{figure}
%
% bibliography
%\section{Reference}
