\chapter{Sources of tracer under wetting and drying}
%

% - Purpose & Problem description:
%     These first two parts give reader short details about the test case,
%     the physical phenomena involved and specify how the numerical solution will be validated
%
\section{Purpose}
%
This test demonstrates the behaviour of buoyant tracer released from a source in an intertidal environment. 
The purposes are to ensure tracer mass conservation (particuarly under wetting and drying), and to 
raise a flag if new code developments have modified the advection or diffusion of tracers.

\section{Description}
%
The model domain is a channel 2000m long and 100m wide, with a sloping bed. Buoyant tracer is added to 
the model from a source at the middle of the domain (unit flow rate and concentration). The tracer rises to the surface and spreads horizontally, as 
the water level rises and falls, inundating and exposing part of the bed. Water level variations are defined using a sine curve at the offshore boundary.
%
% - Reference:
%     This part gives the reference solution we are comparing to and
%     explicits the analytical solution when available;
%
% bibliography can be here or at the end
%\subsection{Reference}
%
%
\subsection{Reference}
%

%
% - Geometry and Mesh:
%     This part describes the mesh used in the computation
%
%
\subsection{Geometry and Mesh}
%
\subsubsection{Bathymetry}
%
Sloped bed with constant gradient from $z = 0$~m to $z = 2$~m.
%
\subsubsection{Geometry}
%
Channel length = 2000~m\\
Channel width = 100~m
%
\subsubsection{Mesh}
%
1000 triangular elements\\
606 nodes\\
6 fixed planes regularly spaced
%
% - Physical parameters:
%     This part specifies the physical parameters
%
%
\subsection{Physical parameters}
%
Constant diffusion of velocity:
\begin{itemize}
\item Horizontal: $10^{-4}$~m$^2$/s,
\item Vertical: 1.0$~m$^2$/s.
\end{itemize}
Constant diffusion of tracer: 
\begin{itemize}
\item Horizontal: 0.01,
\item Vertical: $10^{-6}$~m$^2$/s.
\end{itemize}
Tracer density law specifying a $\beta$ spatial expansion coefficient of
0.0003~K$^{-1}$ and a standard value of the tracer of 0.0.\\
Coriolis: no\\
Wind: no
%
% Experimental results (if needed)
%\subsection{Experimental results}
%
% bibliography can be here or at the end
%\subsection{Reference}
%
% Section for computational options
%\section{Computational options}
%
% - Initial and boundary conditions:
%     This part details both initial and boundary conditions used to simulate the case
%
%
\subsection{Initial and Boundary Conditions}
%
\subsubsection{Initial conditions}
%
Constant water level at $z = 2$~m\\
No velocity\\
Initial value of tracer = 0\\
%
\subsubsection{Boundary conditions}
%
Elevation and tracer prescribed with free velocity at offshore boundary ($z = 0$~m). All other boundaries closed\\
Bottom friction: Nikuradse�s formula with roughness length of 0.001~m\\
Tracer discharge at source: 1.0~m$^3$/s\\
Tracer value at source: 1.0~g/L or kg/m$^3$
%
\subsection{General parameters}
%
Time step: 10~s\\
Simulation duration: 86400~s (1 day)
%
% - Numerical parameters:
%     This part is used to specify the numerical parameters used
%     (adaptive time step, mass-lumping when necessary...)
%
%
\subsection{Numerical parameters}
%
Hydrostatic version\\
Advection of velocities: Characteristics (default scheme)\\
Advection of tracers: Leo Postma scheme for tidal flats\\
Tidal flats enabled with equations solved everywhere and flux limiting.
%
\subsection{Comments}
%
% - Results:
%     We comment in this part the numerical results against the reference ones,
%     giving understanding keys and making assumptions when necessary.
%
%
\section{Results}
%
The buoyant tracer rises and is advected onshore by the currents and rising water levels. 
The water then recedes, exposing the bank and carrying the plume offshore. 
Analysis of the tracer mass shows that tracer is conserved as the bank inundates and then dries. 
The simulation demonstrates that:
\begin{itemize}
\item a source of buoyant tracer behaves as expected (tracer rises to the surface, is advected with the ambient currents, and pools near the source during periods of weak current),
\item the value of tracer on a dry node is retained from the last time the node was wet,
\item tracer mass is conserved under wetting and drying.
\end{itemize}
%
\section{Conclusion}
%
\telemac{3d} simulates correctly the buoyancy of an active tracer, and maintains mass balance in intertidal areas.
%


