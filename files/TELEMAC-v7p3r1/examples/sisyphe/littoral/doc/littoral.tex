\chapter{littoral}
%

% - Purpose & Problem description:
%     These first two parts give reader short details about the test case,
%     the physical phenomena involved and specify how the numerical solution will be validated
%
\section{Purpose}
%
This case test the coupling model between Telemac2dTomawac and sisyphe.
%
\section{Description of the problem}
%
This is the classical test case of a rectilinear beach with sloping bed
The model allows to calculate the littoral transport.
! 
This test case illustrates the effect of waves which is :
\begin{itemize}
\item to generate the current induced littoral current parallel to the beach
\item to increase the sand transport rate using the Bijker sand transport formula.
  \end{itemize}

% - Reference:
%     This part gives the reference solution we are comparing to and
%     explicits the analytical solution when available;
%
%
%\subsection{Reference}
%

% - Physical parameters:
%     This part specifies the geometry, details all the physical parameters
%     used to describe both porous media (soil model in particularly) and
%     solute characteristics (dispersion/diffusion coefficients, soil <=> pollutant interactions...)
%
%
\subsection{Geometry and Mesh}
%
The beach is 1000 m long, 200 m wide
 The beach slope (Y=200m) is 5\% and defined in corfon.f
 The water depth along the open boundary (Y=0) is h=10m
We use a trianglular regular grid 


The mesh is as shown on Figure \ref{littoralmesh}
\begin{figure} [!h]
\centering
\includegraphicsmaybe{[width=0.85\textwidth]}{../img/fond.png}
 \caption{mesh of the case littoral}
\label{littoralmesh}
\end{figure}

% - Initial and boundary conditions:
%     This part details both initial and boundary conditions used to simulate the case
%
%
\subsection{Initial and Boundary Conditions}
%
$\Rightarrow $ Offshore (Y=0): Offshore wave imposed/no littoral current/no set up 

Tomawac:
The wave height is imposed on the offshore boundary (5 4 4) (Hs=1m), for a wave period (Tp=8s).

Telemac2D:
The current and free surface are imposed to 0 along the offshore boundary (5 5 5).

$\Rightarrow $ Left and right hand side of the domain (X=0, X=1000m):recirculation condition 

Tomawac:
The wave height is imposed on the offshore boudary (5 4 4), based on the model solution, calculated  at the center line of the domain.  This is done in limwac.f 

Telemac :
the model solution for the current (4 5 5) on the center line of the model domain are copied on both right and hand side. This is done in bord.f.

%
% - Numerical parameters:
%     This part is used to specify the numerical parameters used
%     (adaptive time step, mass-lumping when necessary...)
%
%
%\subsection{Numerical parameters}
%

% - Results:
%     We comment in this part the numerical results against the reference ones,
%     giving understanding keys and making assumptions when necessary.
%
%
\section{Results}
%
Results (littoral current and transport rates) as well as wave set up/set down are in good agreement with
expectations from theoretical classical results (Longuet Higgins).The model is able to reproduce the wave
induced current, as well as the effect of set down/set up as the waves dissipate in the breaking zone.
The sediment transport rate is located in the near shore breaking zone, where the longshore current is
generated.
Similar results for the littoral transport could be obtained by using an integrated formula (e.g. CERC formula).

The results are presented Figures \ref{resultsT2D} (Velocity U) \ref{resultsTOM}(Wave heigth Hm0) and  \ref{resultsSIS} (Bed Shear stress)
\begin{figure} [!h]
\centering
\includegraphicsmaybe{[width=0.85\textwidth]}{../img/resultsT2D.png}
 \caption{Velocity along U of the case littoral}
\label{resultsT2D}
\end{figure}
\begin{figure} [!h]
\centering
\includegraphicsmaybe{[width=0.85\textwidth]}{../img/resultsTOM.png}
 \caption{Wave heigth Hm0 of the case littoral}
\label{resultsTOM}
\end{figure}
\begin{figure} [!h]
\centering
\includegraphicsmaybe{[width=0.85\textwidth]}{../img/resultsSIS.png}
 \caption{Bed shear Stress of the case littoral}
\label{resultsSIS}
\end{figure}

% Here is an example of how to include the graph generated by validateTELEMAC.py
% They should be in test_case/img
%\begin{figure} [!h]
%\centering
%\includegraphics[scale=0.3]{../img/mygraph.png}
% \caption{mycaption}\label{mylabel}
%\end{figure}


