\chapter{bumpflu}

\section{Description of the problem}
\bigskip
This test case presents a flow over a bump on the bed with subcritical condition.
It allows to show that \telemac{2d} is able to correctly reproduce 
the hydrodynamic impact of a changing bed slopes, vertical flow 
contractions and expansions. Furthermore, It allows to have a good 
representation of flows computed in steady and transient flow regimes. \\
The solution produced by \telemac{2d} in a frictionless channel 
presenting an idealised bump on the bottom  is compared with 
the analytical solution to this problem. For this test case, 
the studied flow regime is sub-critical.

\bigskip
The geometry dimensions of the channel are 2~m wide and 20.5~m long. 
It is horizontal with a 4~m long bump in its middle 
(see Figure \ref{t2d:bumpflu:fig:baty}). The maximum elevation of 
the bump is 0.2~m (see Figure \ref{t2d:bumpflu:fig:profil}) with 
the bottom $z_f$ describes by the following equation :
\begin{equation*}
z_f = \left\{
\begin{array}{rl}
  -0.05(x-10)^2 & \text{ if 8 m < x < 12 m} \\
-0.20 & \text{ elsewhere}
\end{array}
\right.
\end{equation*}
Note that the horizontal viscosity turbulent is constant and equal to zero. 
However instead of prescribing a zero viscosity, no diffusion step could 
have been used (keyword DIFFUSION OF VELOCITY prescribed to NO).
\begin{figure}[!htbp]
 \centering
 \includegraphicsmaybe{[width=\textwidth]}{../img/Mesh_topography.png}
 \caption{Mesh and topography of the channel.}
 \label{t2d:bumpflu:fig:baty}
\end{figure}
\begin{figure}[!htbp]
 \centering
 \includegraphicsmaybe{[width=\textwidth]}{../img/bottom_level.png}
 \caption{Profile of the bump.}
 \label{t2d:bumpflu:fig:profil}
\end{figure}

\section{Initial and boundary conditions}
\bigskip
The initial conditions are a null velocity and an analytical solution 
(Equation \eqref{t2d:bumpflu:eq:ha}) for the water depth.

\bigskip
The boundary conditions are:
\begin{itemize}
\item At the channel entrance, the flow rate is $Q =  8.858893836~\text{m}^3\text{s}^{-1}$  
(note that  $\displaystyle{q=\frac{Q}{B}}$, $B$ being the channel width). 
The flow rate is taken so that the $q$ value is equal at $\sqrt{gh}$ in entrance.
\item At the channel outlet, the free surface elevation is $ z_{s} = 1.8~\text{m}$. 
Therefore, the depth
$\displaystyle{h =  z_{s}-z_{f} = 2~\text{m}}$.
\item No friction is taken into account on the bottom and on the lateral walls.
\end{itemize}

\section{Mesh and numerical parameters}
\bigskip
The mesh is regular, with a higher resolution in the middle of the channel. 
It is made up with quadrangles split into two triangles.
It is composed of 2,620 triangular elements (1,452 nodes) and the size of 
triangles ranges between 0.25~m and 0.5~m. The triangular elements types 
are linear triangles (P1) for velocities and for water depth.

\bigskip
The time step is 0.01~s for a period of 10~s. In fact, \telemac{2d} is run forward 
in time until a steady state flow is obtained. The resolution accuracy 
for the velocity is taken at $10^{-5}$.

\bigskip
Note that for numerical resolution, the conjugate gradient on a normal equation 
is used for solving the propagation step (option 1). Furthermore the treatment 
of linear system is done with a wave equation. To solve advection, 
the conservative PSI scheme is used for the depth (scheme 5) and 
the conservative N scheme is used for the velocities (scheme 4). 
In addition, the treatment of linear system is done with a wave equation. 
To finish, the implicitation coefficient for depth and velocities is equal at 0.6. 

\section{Results}
\bigskip
From an analytical point of view, no friction on the bottom allows to write the Bernoulli 
equation between the entrance of the channel E (where $q_E = 4.43~\text{m}^2\text{s}^{-1}$ 
and $h_E = 2~\text{m}$) and point A of abscissa $x_A$:
\begin{equation*}
  z_{fA}+h_{A}+ \frac{q_{A}^{2}}{2gh_{A}^{2}}=z_{fE}+h_{E}+ 
  \frac{q_{E}^{2}}{2gh_{E}^{2}}
\end{equation*}
Replacing by the values, the equation can be written as following:
\begin{equation*}
  z_{fA}+h_{A}+ \frac{q_{A}^{2}}{2gh_{A}^{2}}=2.25
\end{equation*}
Therefore, water depth $h_A$ is given analytically by:	  
\begin{equation}
h_{A}^{3}+(z_{fA}-2.05)h_{A}^{2}+\frac{q_{A}^{2}}{2g}=0
\label{t2d:bumpflu:eq:ha}
\end{equation}
In this test case, the numerical results are compared with the 
analytical solution, when the state flow is steady 
(Figure \ref{t2d:bumpflu:fig:velo}). Furthermore, the computed 
Froude number is also compared with the analytical solution 
$\displaystyle F_{r}=\frac{q}{h\sqrt{gh}}$.
The solution produced by \telemac{2d} is in close agreement with 
the analytical solution as shown on the Figures \ref{t2d:bumpflu:fig:h} and \ref{t2d:bumpflu:fig:fr}. 

\begin{figure}[!htbp]
 \centering
 \includegraphicsmaybe{[width=\textwidth]}{../img/velocity_vector.png}
 \caption{Velocity field for the steady state flow.}
 \label{t2d:bumpflu:fig:velo}
\end{figure}
\begin{figure}[!htbp]
 \centering
 \includegraphicsmaybe{[width=\textwidth]}{../img/free_surface.png}
 \caption{Comparison between analytical solution and \telemac{2d} 
 solution for the free surface elevation.}
 \label{t2d:bumpflu:fig:h}
\end{figure}
\begin{figure}[!htbp]
 \centering
 \includegraphicsmaybe{[width=\textwidth]}{../img/froude_number.png}
 \caption{Comparison between analytical solution and \telemac{2d} 
 solution for the Froude number.}
 \label{t2d:bumpflu:fig:fr}
\end{figure}

\bigskip
To conclude, this type of channel flow is driven by advection 
and pressure gradient terms. It is adequately reproduced by 
\telemac{2d} in sub-critical flow regime.

