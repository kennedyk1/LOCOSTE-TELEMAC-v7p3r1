\chapter{weirs}

\section{Description of the problem}
\bigskip
This test case presents a flow over three successive sills in 
a rectangular channel treated as singularities. It allows to show 
that \telemac{2d} is able to treat a number of flow singularities 
as internal boundary conditions. Moreover, it allows also to check 
the tracers function for this type of problem in \telemac{2d}.

\bigskip
The sills may be considered under the traditional approach made 
in open channel hydraulics through a relation between two superimposed 
open boundaries, short culverts may be treated as a couple of nodes
 with respective source and sink terms.\\
This sills description, which are represented in a channel as internal 
singularities, was introduced in \telemac{2d} in order to avoid the 
multiplication of computational nodes and associated reduction in time 
step when a sill is represented thanks to variations in the bathymetry.\\
The weir law as traditionally used in channel hydraulics is prescribed 
through two boundary conditions: one upstream the weir and one downstream. 
It should be mentioned however that this option gives satisfactory results 
only if the flow is relatively perpendicular to the weir, which is 
the case in the present test.

\bigskip
In this test case, the sills  are so represented as internal 
singularities, as just indicated.
The geometry dimensions of rectangular channel are 848~m wide 
and  3,522~m long. The channel is flat bottom and it is decomposed 
of four part reaches 848~m long. The three upstream reaches are limited 
at their downstream end by 3 sills with crest heights 1.8~m (upstream sill), 
1.6~m and 1.4~m (downstream sill).\\
Note that the turbulent viscosity is constant and equal to $1~\text{m}^2~\cdot~\text{s}^{-1}$.

\section{Initial and boundary conditions}

\bigskip
The initial conditions are a null velocity, 
a water depth of 1.35~m and a tracer value of 50.

\bigskip
The boundary conditions (Figure \ref{t2d:weirs:fig:geo}) are:
\begin{itemize}
\item At the channel entrance, the flow rate is 
$Q = 600~\text{m}^3\text{s}^{-1}$ and the tracer value is 100.
\item At the channel outlet, the water depth is 
$h=1.35~\text{m}$. 
\item On  bottom friction, the Strickler formula 
with friction coefficient equal to 
30~$\text{m}^{1/3}\cdot\text{s}^{-1}$ is imposed.
\item No friction is taken into account on lateral walls.
\end{itemize}
\begin{figure}[!htbp]
 \centering
 \includegraphicsmaybe{[width=0.7\textwidth]}{../img/geo.png}
 \caption{Mesh and topography of the domain.}
 \label{t2d:weirs:fig:geo}
\end{figure}
\section{Numerical parameters}
\bigskip
The mesh is regular along the domain. It is generally made up 
with quadrangles split into two triangles. It is irregular in 
the upstream reach in order to test the sensitivity on this 
feature of the flow results above the sill (see Figure \ref{t2d:weirs:fig:geo}).
It is composed of 870 triangular elements (519 nodes) and 
the size of triangles ranges between 53~m and 120~m. The triangular 
elements types are linear triangles (P1, 3 values per element, the corners) 
for water depth and quasi-bubble triangle (4 values per element, the corners 
and the element center) for velocities.

\bigskip
The time step is 150~s for a period of 6,000~s. 
The resolution accuracy for the velocity is taken at $10^{-10}$.
 
\bigskip
Note that for numerical resolution, GMRES (Generalized Minimal Residual Method) 
is used for solving the propagation step (option 7). To solve advection, 
the characteristics scheme (scheme 1), and the conservative psi scheme (scheme 5)
is used respectively for the velocities and for the depth. To finish, 
the implicitation coefficient for depth and velocities is equal at 0.55.\\
It should also be noted that a tracer is used. The solver for 
propagation is the conjugate gradient (option 1). For the advection 
resolution, it's used a conservative N-scheme(scheme 4) and the solver 
for diffusion of tracer is also the conjugate gradient (option 1).

\section{Results}
\bigskip
As it can be seen Figure \ref{t2d:weirs:velocity}, 
the velocity field remains regular in the different reaches of the channel.

\bigskip
The water level increases progressively as expected 
in the three upstream reaches during the simulated period 
(see Figure \ref{t2d:weirs:fig:z}). The relations between 
the discharge (per unit of width) on the sill, the water levels 
upstream and downstream and the sill crest elevation must be respected. 
They are:
\begin{itemize}
\item Free overflow weir:
\begin{equation*}
q=\mu \sqrt{2g}\left( z_{up}-z_{sill} \right)^{3/2}
\end{equation*}
\item drowned weir :
 	\begin{equation*}
q=\frac{2}{3\sqrt{3}}C_{d} \sqrt{2g}\left( z_{down}-z_{sill} \right)\sqrt{\left( z_{up}-z_{down} \right)}
\end{equation*}
\end{itemize}
The transition from free overflow to drowned condition is defined by:
\begin{equation}
  	z_{down} \leq z_{sill}+ \frac{2}{3}\left( z_{up}-z_{sill} \right)
  	\label{t2d:weirs:eq:1}
\end{equation}
Where $z_{up}$, $z_{down}$ and $z_{sill}$ are respectively 
water level upstream (m), water level downstream (m) and 
sill crest elevation (m).  $q$ is discharge per unit width 
($\text{m}^2\cdot\text{s}^{-1}$) and $C_{d}$ is the discharge coefficient 
(usually between 0.4 and 0.5).\\
The \telemac{2d} results respect well these relation \eqref{t2d:weirs:eq:1} 
(Figure  \ref{t2d:weirs:fig:z}). Furthermore the tracer propagation is well
 carry out through internal singularities (Figure \ref{t2d:weirs:fig:tracer}).
%
\begin{figure}[!htbp]
\begin{minipage}[t]{0.50\textwidth}
 \centering
 \includegraphicsmaybe{[width=\textwidth]}{../img/figure_velo1500s.png}
\end{minipage}%
\begin{minipage}[t]{0.50\textwidth}
 \centering
 \includegraphicsmaybe{[width=\textwidth]}{../img/figure_velo3000s.png}
\end{minipage}
\begin{minipage}[t]{0.50\textwidth}
 \centering
 \includegraphicsmaybe{[width=\textwidth]}{../img/figure_velo4500s.png}
\end{minipage}%
\begin{minipage}[t]{0.50\textwidth}
 \centering
 \includegraphicsmaybe{[width=\textwidth]}{../img/figure_velo6000s.png}
\end{minipage}
 \caption{Evolution of velocity field in time.}
 \label{t2d:weirs:velocity}
\end{figure}


\begin{figure}[!htbp]
 \centering
 \includegraphicsmaybe{[width=0.85\textwidth]}{../img/free_surface.png}
 \caption{Evolution of the free surface elevation in time.}
 \label{t2d:weirs:fig:z}
\end{figure}

\begin{figure}[!htbp]
 \centering
 \includegraphicsmaybe{[width=0.85\textwidth]}{../img/Tracer.png}
 \caption{Evolution of the tracer in time.}
 \label{t2d:weirs:fig:tracer}
\end{figure}

\bigskip
To conclude, \telemac{2d} computes adequately weir 
flows as given by analytical hydraulic laws.
This type of flow is represented as an internal 
 singularity in the model.
%