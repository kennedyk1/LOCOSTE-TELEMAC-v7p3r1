\chapter{hydraulic jump}


\section{Description of the problem}
\bigskip
This test case presents a flow over a bump on the bed with 
super-critical condition and hydraulic jump.
It allows to show that \telemac{2d} is able to correctly reproduce 
the hydrodynamic impact of a changing bed slopes, vertical flow 
contractions and expansions. Furthermore, It allows to have a good 
representation of flows computed in steady and transient flow regimes. 
Finally, with appropriate choice of the mesh resolution, this test case 
allows also to see that \telemac{2d} produces solutions without wiggles 
(i.e. oscillations in space at a given time or oscillations in time 
at a given position). 

\bigskip
The solution produced by \telemac{2d} in a channel with friction 
presenting an idealised bump on the bottom  is compared with 
the analytical solution to this problem. The studied flow regime is 
sub-critical upstream; a transition to super-critical flow conditions 
is located on the bump, and the downstream water level imposes the presence 
of an hydraulic jump in the downstream reach of the channel.

\bigskip
The geometry dimensions of the channel are 2~m wide and 20.5~m long. 
It is horizontal with a 4 m long bump in its middle 
(see Figure \ref{t2d:hydraulic_jump:fig:baty}). The maximum elevation 
of the bump is 0.2~m (see Figure \ref{t2d:hydraulic_jump:fig:profil}) 
with the bottom $z_f$ is describe by the equation following :
\begin{equation*}
z_f = \left\{
\begin{array}{rl}
-0.05(x-10)^2~\text{m}& \text{ if 8 m < x < 12 m} \\
-0.20~\text{m} & \text{ elsewhere}
\end{array}
\right.
\end{equation*}
Note that the horizontal viscosity turbulent is constant and equal to zero. 
However instead of prescribing a zero viscosity, no diffusion step 
could have been used (keyword DIFFUSION OF VELOCITY prescribed to NO).

\begin{figure}[!htbp]
 \centering
 \includegraphicsmaybe{[width=\textwidth]}{../img/Mesh_topography.png}
 \caption{Mesh and topography of the channel.}
 \label{t2d:hydraulic_jump:fig:baty}
\end{figure}
\begin{figure}[!htbp]
 \centering
 \includegraphicsmaybe{[width=\textwidth]}{../img/bottom_level.png}
 \caption{Profile of the bump.}
 \label{t2d:hydraulic_jump:fig:profil}
\end{figure}
\section{Initial and boundary conditions}
\bigskip
The initial conditions are a null velocity and water level 
equal to 0.4~m along channel.

\bigskip
The boundary conditions are:
\begin{itemize}
\item  At the channel entrance, the flow rate is $Q = 2~\text{m}^3\text{s}^{-1}$.
\item  At the channel outlet, the free surface elevation is 
$z_{s} = 0.4~\text{m}$. Therefore, the depth
$\displaystyle{h = z_{s}-z_{f} = 0.6~\text{m}}$.
\item On the bottom Strickler formula with friction coefficient 
equal to 40~$\text{m}^{1/3}.\text{s}^{-1}$ is imposed.
\item No friction on lateral walls.
\end{itemize}

\section{Mesh and numerical parameters}
\bigskip
The mesh is regular along channel. It is made up with 
quadrangles split into two triangles.
It is composed of 2,620 triangular elements (1,452 nodes) 
and the size of triangles ranges between 0.25~m and 0.5~m. 
The triangular elements types are linear triangles 
(P1, 3 values per element, the corners) for water depth and 
quadratic triangle (6 values per element, the corners and the center of the edges) 
for velocities ("t2d\_hydraulic\_jump\_v2p0.case" steering file).

\bigskip
The time step is 0.02~s for a period of 50~s. In fact, \telemac{2d} is run forward in time until a steady state flow is obtained. The resolution accuracy for the velocity is taken at $10^{-5}$.

\bigskip
Note that for numerical resolution, GMRES (Generalized Minimal Residual Method) 
is used for solving the propagation step (option 7). To solve advection, 
the characteristics scheme is used on velocities (scheme 1) and 
the conservative PSI scheme is used for the depth (scheme 5). 
In addition, the treatment of linear system is done with a wave equation.
To finish, the implicitation coefficient for depth and velocities is equal at 0.6.

\section{Results}
\bigskip
As seen in the test case of the section \ref{t2d:bumpcri:sect}, 
the sub-critical flow upstream the bump and the transition are 
correctly reproduced with \telemac{2d}.\\
The prescribed downstream elevation defines a backwater line which 
takes into account roughness effects on the downstream reach. 
The super-critical flow issuing from the downstream foot of 
the bump also generates a water line. The hydraulic jump should be 
located in a position where the water depths of these two water lines 
are conjugate, i.e. they are related by the hydraulic jump equation:
 \begin{equation*}
 \frac{h_1}{h_2}=\frac{1}{2}\left(\sqrt{1+8 F_{r1}^2}-1\right)
 \end{equation*}
where index $1$ relates to the position upstream the jump and index $2$ 
relates to the position downstream the jump. Let us notice that 
it is necessary to have bottom friction to have a well-posed problem in 
this test case. In the contrary, the water lines issuing from the foot of 
the bump and from the downstream end respectively would be parallel to the bed,
 and the jump could not occur in the downstream reach.

\bigskip
The hydraulic jump computed by \telemac{2d} occurs in a position where 
$h_1=0.18~\text{m}$ , $h_2=0.41~\text{m}$ and $F_{r1} = 1.64$ 
(see Figures \ref{t2d:hydraulic_jump:fig:h} and \ref{t2d:hydraulic_jump:fig:fr}). 
These values satisfy with more or less accuracy the hydraulic jump formula.
In addition, the followings other results are well obtained :
\begin{itemize}
\item The output flow rate is equal to the input flow rate thank to flow 
acceleration after the bump (Figure \ref{t2d:hydraulic_jump:fig:velo}).
\item The Froude number is equal to 1 at the position of the bump 
(Figure \ref{t2d:hydraulic_jump:fig:fr}).
\end{itemize}
%
\begin{figure}[!htbp]
 \centering
 \includegraphicsmaybe{[width=\textwidth]}{../img/free_surface.png}
 \caption{Comparison between analytical solution and \telemac{2d} solution for the free surface elevation.}
 \label{t2d:hydraulic_jump:fig:h}
\end{figure}
\begin{figure}[!htbp]
 \centering
 \includegraphicsmaybe{[width=\textwidth]}{../img/velocity_vector.png}
 \caption{Velocity field for the steady state flow.}
 \label{t2d:hydraulic_jump:fig:velo}
\end{figure}
%
\begin{figure}[!htbp]
 \centering
 \includegraphicsmaybe{[width=\textwidth]}{../img/froude_number.png}
 \caption{Froude number of \telemac{2d}.}
 \label{t2d:hydraulic_jump:fig:fr}
\end{figure}

\bigskip
To conclude, the transition from a super-critical to a sub-critical 
flow regime through a hydraulic jump is properly reproduced by \telemac{2d}, 
provided the mesh resolution is fine enough. The hydraulic jump formula is satisfied 
and the jump is located in the right position.



