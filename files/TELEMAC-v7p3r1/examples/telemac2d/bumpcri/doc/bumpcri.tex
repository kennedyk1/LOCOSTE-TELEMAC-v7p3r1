\chapter{bumpcri}
%
\label{t2d:bumpcri:sect}
%
\section{Description of the problem}
\bigskip
This test case presents a flow over a bump on the bed 
with super-critical condition.
It allows to show that \telemac{2d} is able to correctly 
reproduce the hydrodynamic impact of a changing bed slopes, 
vertical flow contractions and expansions. Furthermore, 
it allows to have a good representation of flows computed in 
steady and transient flow regimes. \\
The solution produced by \telemac{2d} in a frictionless channel 
presenting an idealised bump on the bottom  is compared with the analytical 
solution to this problem.  Flow conditions are such that the bump 
generates a transition from sub-critical (fluvial) to super-critical 
(torrential) flow for this problem.

\bigskip
The geometry dimensions of the channel are 2~m wide and 
20.5~m long. It is horizontal with a 4~m long bump in 
its middle (see Figure \ref{t2d:bumpcri:fig:baty}). 
The maximum elevation of the bump is 0.2~m (see Figure \ref{t2d:bumpcri:fig:profil}) 
with the bottom $z_f$ describes by the following equation :
\begin{equation*}
z_f = \left\{
\begin{array}{rl}
  -0.05(x-10)^2~\text{m}& \text{ if 8 m < x < 12 m} \\
-0.20~\text{m} & \text{ elsewhere}
\end{array}
\right.
\end{equation*}
Note that the horizontal viscosity turbulent is constant and equal to zero. 
However instead of prescribing a zero viscosity, no diffusion step could 
have been used (keyword DIFFUSION OF VELOCITY prescribed to NO).
%
\begin{figure}[!htbp]
 \centering
 \includegraphicsmaybe{[width=\textwidth]}{../img/Mesh_topography.png}
 \caption{Mesh and topography of the channel.}
 \label{t2d:bumpcri:fig:baty}
\end{figure}
\begin{figure}[!htbp]
 \centering
 \includegraphicsmaybe{[width=\textwidth]}{../img/bottom_level.png}
 \caption{Profile of the bump.}
 \label{t2d:bumpcri:fig:profil}
\end{figure}
%
\section{Initial and boundary conditions}
\bigskip
The initial conditions are a null velocity and an analytical solution 
(Equation \eqref{t2d:bumpcri:eq:ha}) for the water depth.

\bigskip
The boundary conditions are:
\begin{itemize}
\item At the channel entrance, the flow rate is $Q = 0.6~\text{m}^3\text{s}^{-1}$ 
(note that  $\displaystyle{q=\frac{Q}{B}}$, $B$ being the channel 
width and the depth is  $\displaystyle{h = z_{s}-z_{f} = 0.5~\text{m}}$, 
with the free surface elevation $z_{s}$).
\item At the channel outlet, the free velocity and free water level are 
imposed on the liquid boundary.
\item No friction is taken into account on the bottom and on the lateral walls.
\end{itemize}

\section{Mesh and numerical parameters}
\bigskip
The mesh is regular, with a higher resolution in the middle of the channel. 
It is made up with quadrangles split into two triangles.
It is composed of 2,620 triangular elements (1,452 nodes) and the size 
of triangles ranges between 0.25~m and 0.5~m. The triangular elements types 
are linear triangles (P1) for velocities and for water depth.

\bigskip
The time step is 0.03~s for a period of 30~s. In fact, \telemac{2d} is 
run forward in time until a steady state flow is obtained. The resolution 
accuracy for the velocity is taken at $10^{-5}$.

\bigskip
Note that for numerical resolution, the conjugate gradient on a normal 
equation is used for solving the propagation step (option 1).
Furthermore the treatment of linear system is done with a wave equation. 
To solve advection, the characteristics scheme is used for the velocities 
(scheme 1) and the conservative PSI scheme is used for the depth (scheme 5, mandatory). 
To finish, the implicitation coefficient for depth and velocities is equal at 1.

\section{Results}

\bigskip
From an analytical point of view, the critical depth is :
\begin{equation*}
h_c=\left(\frac{q_{c}^2}{g}\right)^{1/3}
\end{equation*}
This value is obtained above the bump.
No friction on the bottom thus allows to write the Bernoulli 
equation between the bump and any point A of abscissa $x_A$:
\begin{equation*}
  z_{fA}+h_{A}+ \frac{q_{A}^{2}}{2gh_{A}^{2}}=z_{fc}+E_{c}
\end{equation*}
where the corresponding specific water depth energy $E_c$ is:
\begin{equation*}
E_c=h_c+\frac{q_{c}^2}{2gh_c^2}=\frac{3}{2}h_c
\end{equation*}
Therefore, water depth $h_A$ is given analytically by:	
\begin{equation}
h_A^3+(z_A-z_c-E_c)h_A^2+\frac{q_{A}^2}{2g}=0
\label{t2d:bumpcri:eq:ha}
\end{equation} 

\bigskip
At the foot of the bump, positive solutions are $h_{upstream}=0.49535~\text{m}$ 
and $h_{downstream}=0.49535~\text{m}$. These solutions are respectively 
the upstream sub-critical and downstream super-critical water depths. 
Corresponding free surface elevations are $ z_{s~upstream} = 0.29535~\text{m}$ 
and $z_{s~downstream} = - 0.094~\text{m}$.

\begin{figure}[!htbp]
 \centering
 \includegraphicsmaybe{[width=\textwidth]}{../img/velocity_vector.png}
 \caption{Velocity field for the steady state flow.}
 \label{t2d:bumpcri:fig:velo}
\end{figure}

\bigskip
Numerical results are compared with the analytical solution, when the 
state flow is steady (Figure \ref{t2d:bumpflu:fig:velo}). 
Furthermore, the computed Froude number is also compared with the 
analytical solution $\displaystyle F_{r}=\frac{q}{h\sqrt{gh}}$.
The solution produced by \telemac{2d} is in close agreement with 
the analytical solution as shown on the Figures \ref{t2d:bumpcri:fig:h} 
and \ref{t2d:bumpcri:fig:fr}. 

\begin{figure}[!htbp]
 \centering
 \includegraphicsmaybe{[width=\textwidth]}{../img/free_surface.png}
 \caption{Comparison between analytical solution and \telemac{2d} 
 solution for the free surface elevation.}
 \label{t2d:bumpcri:fig:h}
\end{figure}
\begin{figure}[!htbp]
 \centering
 \includegraphicsmaybe{[width=\textwidth]}{../img/froude_number.png}
 \caption{Comparison between analytical solution and \telemac{2d} 
 solution for the Froude number.}
 \label{t2d:bumpcri:fig:fr}
\end{figure}

\bigskip
To conclude, this transition from a sub-critical to a super-critical flow 
regime over a bump in a channel without friction is adequately reproduced by 
\telemac{2d}, provided the mesh resolution is fine enough. This transition 
occurs at critical depth as announced by channel flow hydraulics.
